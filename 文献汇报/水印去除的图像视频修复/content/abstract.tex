\IEEEtitleabstractindextext{
\begin{abstract}
\justifying 可见水印在图像版权保护中发挥着重要作用,水印的鲁棒性对于抵抗攻击被证明是至关重要的。水印去除技术以对抗性的方式增强可见水印的鲁棒性,因此越来越受到研究者的关注。现有的基于深度学习的水印去除方法以 U-Net 网络架构为主流,并通过编码器和解码器之间的跳连接将浅层纹理特征与深层语义特征结合,以保留更多的图像细节;或辅助以生成对抗的训练方式,使得模型生成更加真实的图像细节。然而,针对水印去除的图像去除修复方法相对较少,更多的研究工作聚焦于通用场景的图像和视频修复,而这些技术在现有的水印去除工作中也被用于专门的水印去除模块或分阶段的细化模块。此外,同样用于生成任务的扩散模型虽然尚未被用于专门的水印去除,但已有较多研究致力于通用的图像修复任务及生成内容的水印嵌入及溯源取证,这也是水印去除作为攻击手段而被研究的目的,即服务于更加鲁棒的水印嵌入技术的设计。本文以水印去除技术的发展脉络为主要线索,对水印去除的图像视频修复技术进行介绍。
\end{abstract}

\begin{IEEEkeywords}
水印去除、水印嵌入、图像视频修复、U-Net、生成对抗网络、生成模型
\end{IEEEkeywords}}