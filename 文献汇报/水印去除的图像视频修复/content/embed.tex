\section{水印嵌入}
\label{sec:embed}

\subsection{视频水印嵌入}

如前文所述,水印去除作为一种攻击手段,可以为开发更加鲁棒的水印嵌入技术提供线索。Ye等~\cite{ye2022deformable}设计了从输入视频中提取动态场景中持久元素的方法,将每个场景元素表示为可变形的精灵(Deformable Sprite),由三部分组成:1)贯穿整个视频的2D纹理图像,2)元素的逐帧掩码,3)将纹理图像映射到每个视频帧的非刚性形变。得到的分解结果可以用于一致的视频编辑等应用,例如将水印根据纹理和形变动态地嵌入到视频中的对象,可以实现更好的嵌入效果,同时相较于静态水印给水印的去除带来更大的难度。

\subsection{扩散模型和水印}
可以看到,现有工作多依赖于GAN网络来恢复更加真实的水印去除图像,并取得了相当不错的性能表现。同样作为生成模型,扩散模型可以在相同任务上生成更加真实的图像。尽管尚未出现基于扩散模型的针对水印去除任务的研究,但通用的图像修复模型~\cite{yildirim2023inst,rombach2022high}已如雨后春笋般涌现,一般而言,这些模型也适用于水印去除的图像修复。

除此以外,以水印去除作为攻击手段,研究人员旨在借此设计更加鲁棒的水印嵌入方法。而以扩散模型为代表的生成模型在应用于各种领域的同时,也引发了关于负责任部署的道德关注。因此,将图像水印和扩散模型相结合,让所生成的图像嵌入不可见的水印,以便未来进行检测或识别\cite{fernandez2023stable},也已成为研究的热点。