\section{简介}
\label{sec:introduction}

随着社交媒体的激增,图像和视频成为最流行的记录和传递信息的载体。近年来,数字水印技术已被广泛用于防止图像、视频等多媒体内容非法复制或盗窃的有效解决方案。根据嵌入载体图像中水印数据的可见性将水印技术分为可见和不可见两类。通常,不可见水印适用于作为大多数形式的数字内容知识产权保护机制。除非使用特定的水印提取技术,否则用户无法从视觉感知上区分不可见水印内容和原始载体内容。通过检查可疑内容中是否存在水印,以被动的方式保护内容提供者或作者的版权。而可见水印用于保护必须出于某些目的而发布的数字图像或视频,例如在远程学习网站或数字图书馆中使用的内容,且非法复制是被禁止的。可见水印以更积极主动的方式保护知识产权——可见水印内容通常包含可识别但不显眼的版权图案,用于指示知识产权所有者的身份。除非水印图案能够在不破坏所保护内容的视觉质量的情况下被完全去除,否则没有人可以直接使用带有可见水印的数据。因此,为了满足用户高质量的视觉观感需求,有效的水印去除的图像视频修复技术是必要的。同时,可见水印的嵌入也可能会导致模型学习到错误的特征,进而抑制下游任务的准确性,故需要通过对图像和视频中的水印进行去除和修复以实现更高精度的识别或定位。除此以外,水印去除也可以看作是对水印嵌入图像的一种攻击,而研究如何有效地去除可见水印也为发明更鲁棒的水印嵌入技术提供了线索。

可见水印的嵌入一般是通过将水印图案与载体图像通过 $\alpha$ 混合叠加在一起的。一张带水印的图像$X$通常是通过将水印$W$叠加到载体图像$Y$上获得的。在水印嵌入区域中,水印像素$X(p)$和无水印像素$Y(p)$之间的关系可以表示为:

\begin{equation}
X(p)=\alpha(p) W(p)+(1-\alpha(p)) Y(p)
\end{equation}

其中,$p=(i,j)$ 表示图像中的像素位置,$\alpha(p)$ 是空间变化的不透明度,即图像处理中使用的$\alpha$通道。最常用的水印是半透明的,以保持载体图像内容部分可见,这意味着对于所有像素,$\alpha(p)$的取值范围在$[0, 1]$之间。如果$\alpha(p)$在所有位置上都等于1,那么X就变成了$W$;而如果$\alpha(p)$在所有位置上都等于0,那么$X$就等于$Y$。

水印去除旨在从带水印的图像X中获取无水印的图像$Y$。在给定$W$和$\alpha$的条件下,可以通过逐像素的反向进行操作,简单地逆向合成水印图像的过程

\begin{equation}
Y(p)=\frac{X(p)-\alpha(p) W(p)}{1-\alpha(p)}
\end{equation}

而无水印区域在$X$和$Y$之间保持不变。这一想法不仅适用于传统方法,在给定水印嵌入先验信息的条件下,也可以为水印去除的图像视频修复提供用于指导的边信息。
