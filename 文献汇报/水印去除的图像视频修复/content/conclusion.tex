\section{结论}
\label{sec:conclusion}

水印在日常生活中随处可见,它是一种保护图像视频版权的机制,防止未经许可或授权的使用。在很多的情况下,人们可能希望去除这些水印图像以获得不被遮盖的原始图像。然而,由于水印可以覆盖在各种大小、形状、颜色和透明度的背景图像上的任意位置,同时其通常包含复杂的图案,如扭曲的符号、细线、阴影效果,这使得去除这些水印图像并恢复原始图像是一项具有挑战性的任务。如果没有人工指导或关于载体图像的假设,将很难检测到水印图像并重建原始图像。近年来,随着计算机硬件的不断发展,计算机的计算能力也显著提高,从而推动了深度学习方法的快速进步并产生了一系列基于卷积神经网络的技术。从研究发展历程来看,水印去除既可以采取通用的目标检测模型和图像修复模型,又可以根据任务特性设计专门的模块和网络架构重建视觉效果更好图像。在过去,U-Net、GAN网络等模型架构在水印去除任务上一直作为主流的网络模型并取得了很好的性能表现。放眼未来,我们也期待有更多的研究可以基于现如今性能更加强大的网络设计(如 Transformer、扩散模型等)以实现更好的水印去除和图像视频修复效果。在这个繁荣的信息时代,水印去除技术的发展也在不断推进水印嵌入技术的进步,这对于网络信息安全领域具有重要意义。