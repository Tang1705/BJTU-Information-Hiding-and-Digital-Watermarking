\documentclass[11pt]{article}
	\usepackage{ctex}
    \usepackage[breakable]{tcolorbox}
    \usepackage{parskip} % Stop auto-indenting (to mimic markdown behaviour)
    

    % Basic figure setup, for now with no caption control since it's done
    % automatically by Pandoc (which extracts ![](path) syntax from Markdown).
    \usepackage{graphicx}
    % Maintain compatibility with old templates. Remove in nbconvert 6.0
    \let\Oldincludegraphics\includegraphics
    % Ensure that by default, figures have no caption (until we provide a
    % proper Figure object with a Caption API and a way to capture that
    % in the conversion process - todo).
    \usepackage{caption}
    \DeclareCaptionFormat{nocaption}{}
    \captionsetup{format=nocaption,aboveskip=0pt,belowskip=0pt}

    \usepackage{float}
    \floatplacement{figure}{H} % forces figures to be placed at the correct location
    \usepackage{xcolor} % Allow colors to be defined
    \usepackage{enumerate} % Needed for markdown enumerations to work
    \usepackage{geometry} % Used to adjust the document margins
    \usepackage{amsmath} % Equations
    \usepackage{amssymb} % Equations
    \usepackage{textcomp} % defines textquotesingle
    % Hack from http://tex.stackexchange.com/a/47451/13684:
    \AtBeginDocument{%
        \def\PYZsq{\textquotesingle}% Upright quotes in Pygmentized code
    }
    \usepackage{upquote} % Upright quotes for verbatim code
    \usepackage{eurosym} % defines \euro

    \usepackage{iftex}
    \ifPDFTeX
        \usepackage[T1]{fontenc}
        \IfFileExists{alphabeta.sty}{
              \usepackage{alphabeta}
          }{
              \usepackage[mathletters]{ucs}
              \usepackage[utf8x]{inputenc}
          }
    \else
        \usepackage{fontspec}
        \usepackage{unicode-math}
    \fi

    \usepackage{fancyvrb} % verbatim replacement that allows latex
    \usepackage{grffile} % extends the file name processing of package graphics
                         % to support a larger range
    \makeatletter % fix for old versions of grffile with XeLaTeX
    \@ifpackagelater{grffile}{2019/11/01}
    {
      % Do nothing on new versions
    }
    {
      \def\Gread@@xetex#1{%
        \IfFileExists{"\Gin@base".bb}%
        {\Gread@eps{\Gin@base.bb}}%
        {\Gread@@xetex@aux#1}%
      }
    }
    \makeatother
    \usepackage[Export]{adjustbox} % Used to constrain images to a maximum size
    \adjustboxset{max size={0.9\linewidth}{0.9\paperheight}}

    % The hyperref package gives us a pdf with properly built
    % internal navigation ('pdf bookmarks' for the table of contents,
    % internal cross-reference links, web links for URLs, etc.)
    \usepackage{hyperref}
    % The default LaTeX title has an obnoxious amount of whitespace. By default,
    % titling removes some of it. It also provides customization options.
    \usepackage{titling}
    \usepackage{longtable} % longtable support required by pandoc >1.10
    \usepackage{booktabs}  % table support for pandoc > 1.12.2
    \usepackage{array}     % table support for pandoc >= 2.11.3
    \usepackage{calc}      % table minipage width calculation for pandoc >= 2.11.1
    \usepackage[inline]{enumitem} % IRkernel/repr support (it uses the enumerate* environment)
    \usepackage[normalem]{ulem} % ulem is needed to support strikethroughs (\sout)
                                % normalem makes italics be italics, not underlines
    \usepackage{soul}      % strikethrough (\st) support for pandoc >= 3.0.0
    \usepackage{mathrsfs}
    

    
    % Colors for the hyperref package
    \definecolor{urlcolor}{rgb}{0,.145,.698}
    \definecolor{linkcolor}{rgb}{.71,0.21,0.01}
    \definecolor{citecolor}{rgb}{.12,.54,.11}

    % ANSI colors
    \definecolor{ansi-black}{HTML}{3E424D}
    \definecolor{ansi-black-intense}{HTML}{282C36}
    \definecolor{ansi-red}{HTML}{E75C58}
    \definecolor{ansi-red-intense}{HTML}{B22B31}
    \definecolor{ansi-green}{HTML}{00A250}
    \definecolor{ansi-green-intense}{HTML}{007427}
    \definecolor{ansi-yellow}{HTML}{DDB62B}
    \definecolor{ansi-yellow-intense}{HTML}{B27D12}
    \definecolor{ansi-blue}{HTML}{208FFB}
    \definecolor{ansi-blue-intense}{HTML}{0065CA}
    \definecolor{ansi-magenta}{HTML}{D160C4}
    \definecolor{ansi-magenta-intense}{HTML}{A03196}
    \definecolor{ansi-cyan}{HTML}{60C6C8}
    \definecolor{ansi-cyan-intense}{HTML}{258F8F}
    \definecolor{ansi-white}{HTML}{C5C1B4}
    \definecolor{ansi-white-intense}{HTML}{A1A6B2}
    \definecolor{ansi-default-inverse-fg}{HTML}{FFFFFF}
    \definecolor{ansi-default-inverse-bg}{HTML}{000000}

    % common color for the border for error outputs.
    \definecolor{outerrorbackground}{HTML}{FFDFDF}

    % commands and environments needed by pandoc snippets
    % extracted from the output of `pandoc -s`
    \providecommand{\tightlist}{%
      \setlength{\itemsep}{0pt}\setlength{\parskip}{0pt}}
    \DefineVerbatimEnvironment{Highlighting}{Verbatim}{commandchars=\\\{\}}
    % Add ',fontsize=\small' for more characters per line
    \newenvironment{Shaded}{}{}
    \newcommand{\KeywordTok}[1]{\textcolor[rgb]{0.00,0.44,0.13}{\textbf{{#1}}}}
    \newcommand{\DataTypeTok}[1]{\textcolor[rgb]{0.56,0.13,0.00}{{#1}}}
    \newcommand{\DecValTok}[1]{\textcolor[rgb]{0.25,0.63,0.44}{{#1}}}
    \newcommand{\BaseNTok}[1]{\textcolor[rgb]{0.25,0.63,0.44}{{#1}}}
    \newcommand{\FloatTok}[1]{\textcolor[rgb]{0.25,0.63,0.44}{{#1}}}
    \newcommand{\CharTok}[1]{\textcolor[rgb]{0.25,0.44,0.63}{{#1}}}
    \newcommand{\StringTok}[1]{\textcolor[rgb]{0.25,0.44,0.63}{{#1}}}
    \newcommand{\CommentTok}[1]{\textcolor[rgb]{0.38,0.63,0.69}{\textit{{#1}}}}
    \newcommand{\OtherTok}[1]{\textcolor[rgb]{0.00,0.44,0.13}{{#1}}}
    \newcommand{\AlertTok}[1]{\textcolor[rgb]{1.00,0.00,0.00}{\textbf{{#1}}}}
    \newcommand{\FunctionTok}[1]{\textcolor[rgb]{0.02,0.16,0.49}{{#1}}}
    \newcommand{\RegionMarkerTok}[1]{{#1}}
    \newcommand{\ErrorTok}[1]{\textcolor[rgb]{1.00,0.00,0.00}{\textbf{{#1}}}}
    \newcommand{\NormalTok}[1]{{#1}}

    % Additional commands for more recent versions of Pandoc
    \newcommand{\ConstantTok}[1]{\textcolor[rgb]{0.53,0.00,0.00}{{#1}}}
    \newcommand{\SpecialCharTok}[1]{\textcolor[rgb]{0.25,0.44,0.63}{{#1}}}
    \newcommand{\VerbatimStringTok}[1]{\textcolor[rgb]{0.25,0.44,0.63}{{#1}}}
    \newcommand{\SpecialStringTok}[1]{\textcolor[rgb]{0.73,0.40,0.53}{{#1}}}
    \newcommand{\ImportTok}[1]{{#1}}
    \newcommand{\DocumentationTok}[1]{\textcolor[rgb]{0.73,0.13,0.13}{\textit{{#1}}}}
    \newcommand{\AnnotationTok}[1]{\textcolor[rgb]{0.38,0.63,0.69}{\textbf{\textit{{#1}}}}}
    \newcommand{\CommentVarTok}[1]{\textcolor[rgb]{0.38,0.63,0.69}{\textbf{\textit{{#1}}}}}
    \newcommand{\VariableTok}[1]{\textcolor[rgb]{0.10,0.09,0.49}{{#1}}}
    \newcommand{\ControlFlowTok}[1]{\textcolor[rgb]{0.00,0.44,0.13}{\textbf{{#1}}}}
    \newcommand{\OperatorTok}[1]{\textcolor[rgb]{0.40,0.40,0.40}{{#1}}}
    \newcommand{\BuiltInTok}[1]{{#1}}
    \newcommand{\ExtensionTok}[1]{{#1}}
    \newcommand{\PreprocessorTok}[1]{\textcolor[rgb]{0.74,0.48,0.00}{{#1}}}
    \newcommand{\AttributeTok}[1]{\textcolor[rgb]{0.49,0.56,0.16}{{#1}}}
    \newcommand{\InformationTok}[1]{\textcolor[rgb]{0.38,0.63,0.69}{\textbf{\textit{{#1}}}}}
    \newcommand{\WarningTok}[1]{\textcolor[rgb]{0.38,0.63,0.69}{\textbf{\textit{{#1}}}}}


    % Define a nice break command that doesn't care if a line doesn't already
    % exist.
    \def\br{\hspace*{\fill} \\* }
    % Math Jax compatibility definitions
    \def\gt{>}
    \def\lt{<}
    \let\Oldtex\TeX
    \let\Oldlatex\LaTeX
    \renewcommand{\TeX}{\textrm{\Oldtex}}
    \renewcommand{\LaTeX}{\textrm{\Oldlatex}}
    % Document parameters
    % Document title
    \title{信息隐藏与数字水印实验}
    \author{唐麒 21120299}
    \date{}
    
    
    
    
    
    
% Pygments definitions
\makeatletter
\def\PY@reset{\let\PY@it=\relax \let\PY@bf=\relax%
    \let\PY@ul=\relax \let\PY@tc=\relax%
    \let\PY@bc=\relax \let\PY@ff=\relax}
\def\PY@tok#1{\csname PY@tok@#1\endcsname}
\def\PY@toks#1+{\ifx\relax#1\empty\else%
    \PY@tok{#1}\expandafter\PY@toks\fi}
\def\PY@do#1{\PY@bc{\PY@tc{\PY@ul{%
    \PY@it{\PY@bf{\PY@ff{#1}}}}}}}
\def\PY#1#2{\PY@reset\PY@toks#1+\relax+\PY@do{#2}}

\@namedef{PY@tok@w}{\def\PY@tc##1{\textcolor[rgb]{0.73,0.73,0.73}{##1}}}
\@namedef{PY@tok@c}{\let\PY@it=\textit\def\PY@tc##1{\textcolor[rgb]{0.24,0.48,0.48}{##1}}}
\@namedef{PY@tok@cp}{\def\PY@tc##1{\textcolor[rgb]{0.61,0.40,0.00}{##1}}}
\@namedef{PY@tok@k}{\let\PY@bf=\textbf\def\PY@tc##1{\textcolor[rgb]{0.00,0.50,0.00}{##1}}}
\@namedef{PY@tok@kp}{\def\PY@tc##1{\textcolor[rgb]{0.00,0.50,0.00}{##1}}}
\@namedef{PY@tok@kt}{\def\PY@tc##1{\textcolor[rgb]{0.69,0.00,0.25}{##1}}}
\@namedef{PY@tok@o}{\def\PY@tc##1{\textcolor[rgb]{0.40,0.40,0.40}{##1}}}
\@namedef{PY@tok@ow}{\let\PY@bf=\textbf\def\PY@tc##1{\textcolor[rgb]{0.67,0.13,1.00}{##1}}}
\@namedef{PY@tok@nb}{\def\PY@tc##1{\textcolor[rgb]{0.00,0.50,0.00}{##1}}}
\@namedef{PY@tok@nf}{\def\PY@tc##1{\textcolor[rgb]{0.00,0.00,1.00}{##1}}}
\@namedef{PY@tok@nc}{\let\PY@bf=\textbf\def\PY@tc##1{\textcolor[rgb]{0.00,0.00,1.00}{##1}}}
\@namedef{PY@tok@nn}{\let\PY@bf=\textbf\def\PY@tc##1{\textcolor[rgb]{0.00,0.00,1.00}{##1}}}
\@namedef{PY@tok@ne}{\let\PY@bf=\textbf\def\PY@tc##1{\textcolor[rgb]{0.80,0.25,0.22}{##1}}}
\@namedef{PY@tok@nv}{\def\PY@tc##1{\textcolor[rgb]{0.10,0.09,0.49}{##1}}}
\@namedef{PY@tok@no}{\def\PY@tc##1{\textcolor[rgb]{0.53,0.00,0.00}{##1}}}
\@namedef{PY@tok@nl}{\def\PY@tc##1{\textcolor[rgb]{0.46,0.46,0.00}{##1}}}
\@namedef{PY@tok@ni}{\let\PY@bf=\textbf\def\PY@tc##1{\textcolor[rgb]{0.44,0.44,0.44}{##1}}}
\@namedef{PY@tok@na}{\def\PY@tc##1{\textcolor[rgb]{0.41,0.47,0.13}{##1}}}
\@namedef{PY@tok@nt}{\let\PY@bf=\textbf\def\PY@tc##1{\textcolor[rgb]{0.00,0.50,0.00}{##1}}}
\@namedef{PY@tok@nd}{\def\PY@tc##1{\textcolor[rgb]{0.67,0.13,1.00}{##1}}}
\@namedef{PY@tok@s}{\def\PY@tc##1{\textcolor[rgb]{0.73,0.13,0.13}{##1}}}
\@namedef{PY@tok@sd}{\let\PY@it=\textit\def\PY@tc##1{\textcolor[rgb]{0.73,0.13,0.13}{##1}}}
\@namedef{PY@tok@si}{\let\PY@bf=\textbf\def\PY@tc##1{\textcolor[rgb]{0.64,0.35,0.47}{##1}}}
\@namedef{PY@tok@se}{\let\PY@bf=\textbf\def\PY@tc##1{\textcolor[rgb]{0.67,0.36,0.12}{##1}}}
\@namedef{PY@tok@sr}{\def\PY@tc##1{\textcolor[rgb]{0.64,0.35,0.47}{##1}}}
\@namedef{PY@tok@ss}{\def\PY@tc##1{\textcolor[rgb]{0.10,0.09,0.49}{##1}}}
\@namedef{PY@tok@sx}{\def\PY@tc##1{\textcolor[rgb]{0.00,0.50,0.00}{##1}}}
\@namedef{PY@tok@m}{\def\PY@tc##1{\textcolor[rgb]{0.40,0.40,0.40}{##1}}}
\@namedef{PY@tok@gh}{\let\PY@bf=\textbf\def\PY@tc##1{\textcolor[rgb]{0.00,0.00,0.50}{##1}}}
\@namedef{PY@tok@gu}{\let\PY@bf=\textbf\def\PY@tc##1{\textcolor[rgb]{0.50,0.00,0.50}{##1}}}
\@namedef{PY@tok@gd}{\def\PY@tc##1{\textcolor[rgb]{0.63,0.00,0.00}{##1}}}
\@namedef{PY@tok@gi}{\def\PY@tc##1{\textcolor[rgb]{0.00,0.52,0.00}{##1}}}
\@namedef{PY@tok@gr}{\def\PY@tc##1{\textcolor[rgb]{0.89,0.00,0.00}{##1}}}
\@namedef{PY@tok@ge}{\let\PY@it=\textit}
\@namedef{PY@tok@gs}{\let\PY@bf=\textbf}
\@namedef{PY@tok@gp}{\let\PY@bf=\textbf\def\PY@tc##1{\textcolor[rgb]{0.00,0.00,0.50}{##1}}}
\@namedef{PY@tok@go}{\def\PY@tc##1{\textcolor[rgb]{0.44,0.44,0.44}{##1}}}
\@namedef{PY@tok@gt}{\def\PY@tc##1{\textcolor[rgb]{0.00,0.27,0.87}{##1}}}
\@namedef{PY@tok@err}{\def\PY@bc##1{{\setlength{\fboxsep}{\string -\fboxrule}\fcolorbox[rgb]{1.00,0.00,0.00}{1,1,1}{\strut ##1}}}}
\@namedef{PY@tok@kc}{\let\PY@bf=\textbf\def\PY@tc##1{\textcolor[rgb]{0.00,0.50,0.00}{##1}}}
\@namedef{PY@tok@kd}{\let\PY@bf=\textbf\def\PY@tc##1{\textcolor[rgb]{0.00,0.50,0.00}{##1}}}
\@namedef{PY@tok@kn}{\let\PY@bf=\textbf\def\PY@tc##1{\textcolor[rgb]{0.00,0.50,0.00}{##1}}}
\@namedef{PY@tok@kr}{\let\PY@bf=\textbf\def\PY@tc##1{\textcolor[rgb]{0.00,0.50,0.00}{##1}}}
\@namedef{PY@tok@bp}{\def\PY@tc##1{\textcolor[rgb]{0.00,0.50,0.00}{##1}}}
\@namedef{PY@tok@fm}{\def\PY@tc##1{\textcolor[rgb]{0.00,0.00,1.00}{##1}}}
\@namedef{PY@tok@vc}{\def\PY@tc##1{\textcolor[rgb]{0.10,0.09,0.49}{##1}}}
\@namedef{PY@tok@vg}{\def\PY@tc##1{\textcolor[rgb]{0.10,0.09,0.49}{##1}}}
\@namedef{PY@tok@vi}{\def\PY@tc##1{\textcolor[rgb]{0.10,0.09,0.49}{##1}}}
\@namedef{PY@tok@vm}{\def\PY@tc##1{\textcolor[rgb]{0.10,0.09,0.49}{##1}}}
\@namedef{PY@tok@sa}{\def\PY@tc##1{\textcolor[rgb]{0.73,0.13,0.13}{##1}}}
\@namedef{PY@tok@sb}{\def\PY@tc##1{\textcolor[rgb]{0.73,0.13,0.13}{##1}}}
\@namedef{PY@tok@sc}{\def\PY@tc##1{\textcolor[rgb]{0.73,0.13,0.13}{##1}}}
\@namedef{PY@tok@dl}{\def\PY@tc##1{\textcolor[rgb]{0.73,0.13,0.13}{##1}}}
\@namedef{PY@tok@s2}{\def\PY@tc##1{\textcolor[rgb]{0.73,0.13,0.13}{##1}}}
\@namedef{PY@tok@sh}{\def\PY@tc##1{\textcolor[rgb]{0.73,0.13,0.13}{##1}}}
\@namedef{PY@tok@s1}{\def\PY@tc##1{\textcolor[rgb]{0.73,0.13,0.13}{##1}}}
\@namedef{PY@tok@mb}{\def\PY@tc##1{\textcolor[rgb]{0.40,0.40,0.40}{##1}}}
\@namedef{PY@tok@mf}{\def\PY@tc##1{\textcolor[rgb]{0.40,0.40,0.40}{##1}}}
\@namedef{PY@tok@mh}{\def\PY@tc##1{\textcolor[rgb]{0.40,0.40,0.40}{##1}}}
\@namedef{PY@tok@mi}{\def\PY@tc##1{\textcolor[rgb]{0.40,0.40,0.40}{##1}}}
\@namedef{PY@tok@il}{\def\PY@tc##1{\textcolor[rgb]{0.40,0.40,0.40}{##1}}}
\@namedef{PY@tok@mo}{\def\PY@tc##1{\textcolor[rgb]{0.40,0.40,0.40}{##1}}}
\@namedef{PY@tok@ch}{\let\PY@it=\textit\def\PY@tc##1{\textcolor[rgb]{0.24,0.48,0.48}{##1}}}
\@namedef{PY@tok@cm}{\let\PY@it=\textit\def\PY@tc##1{\textcolor[rgb]{0.24,0.48,0.48}{##1}}}
\@namedef{PY@tok@cpf}{\let\PY@it=\textit\def\PY@tc##1{\textcolor[rgb]{0.24,0.48,0.48}{##1}}}
\@namedef{PY@tok@c1}{\let\PY@it=\textit\def\PY@tc##1{\textcolor[rgb]{0.24,0.48,0.48}{##1}}}
\@namedef{PY@tok@cs}{\let\PY@it=\textit\def\PY@tc##1{\textcolor[rgb]{0.24,0.48,0.48}{##1}}}

\def\PYZbs{\char`\\}
\def\PYZus{\char`\_}
\def\PYZob{\char`\{}
\def\PYZcb{\char`\}}
\def\PYZca{\char`\^}
\def\PYZam{\char`\&}
\def\PYZlt{\char`\<}
\def\PYZgt{\char`\>}
\def\PYZsh{\char`\#}
\def\PYZpc{\char`\%}
\def\PYZdl{\char`\$}
\def\PYZhy{\char`\-}
\def\PYZsq{\char`\'}
\def\PYZdq{\char`\"}
\def\PYZti{\char`\~}
% for compatibility with earlier versions
\def\PYZat{@}
\def\PYZlb{[}
\def\PYZrb{]}
\makeatother


    % For linebreaks inside Verbatim environment from package fancyvrb.
    \makeatletter
        \newbox\Wrappedcontinuationbox
        \newbox\Wrappedvisiblespacebox
        \newcommand*\Wrappedvisiblespace {\textcolor{red}{\textvisiblespace}}
        \newcommand*\Wrappedcontinuationsymbol {\textcolor{red}{\llap{\tiny$\m@th\hookrightarrow$}}}
        \newcommand*\Wrappedcontinuationindent {3ex }
        \newcommand*\Wrappedafterbreak {\kern\Wrappedcontinuationindent\copy\Wrappedcontinuationbox}
        % Take advantage of the already applied Pygments mark-up to insert
        % potential linebreaks for TeX processing.
        %        {, <, #, %, $, ' and ": go to next line.
        %        _, }, ^, &, >, - and ~: stay at end of broken line.
        % Use of \textquotesingle for straight quote.
        \newcommand*\Wrappedbreaksatspecials {%
            \def\PYGZus{\discretionary{\char`\_}{\Wrappedafterbreak}{\char`\_}}%
            \def\PYGZob{\discretionary{}{\Wrappedafterbreak\char`\{}{\char`\{}}%
            \def\PYGZcb{\discretionary{\char`\}}{\Wrappedafterbreak}{\char`\}}}%
            \def\PYGZca{\discretionary{\char`\^}{\Wrappedafterbreak}{\char`\^}}%
            \def\PYGZam{\discretionary{\char`\&}{\Wrappedafterbreak}{\char`\&}}%
            \def\PYGZlt{\discretionary{}{\Wrappedafterbreak\char`\<}{\char`\<}}%
            \def\PYGZgt{\discretionary{\char`\>}{\Wrappedafterbreak}{\char`\>}}%
            \def\PYGZsh{\discretionary{}{\Wrappedafterbreak\char`\#}{\char`\#}}%
            \def\PYGZpc{\discretionary{}{\Wrappedafterbreak\char`\%}{\char`\%}}%
            \def\PYGZdl{\discretionary{}{\Wrappedafterbreak\char`\$}{\char`\$}}%
            \def\PYGZhy{\discretionary{\char`\-}{\Wrappedafterbreak}{\char`\-}}%
            \def\PYGZsq{\discretionary{}{\Wrappedafterbreak\textquotesingle}{\textquotesingle}}%
            \def\PYGZdq{\discretionary{}{\Wrappedafterbreak\char`\"}{\char`\"}}%
            \def\PYGZti{\discretionary{\char`\~}{\Wrappedafterbreak}{\char`\~}}%
        }
        % Some characters . , ; ? ! / are not pygmentized.
        % This macro makes them "active" and they will insert potential linebreaks
        \newcommand*\Wrappedbreaksatpunct {%
            \lccode`\~`\.\lowercase{\def~}{\discretionary{\hbox{\char`\.}}{\Wrappedafterbreak}{\hbox{\char`\.}}}%
            \lccode`\~`\,\lowercase{\def~}{\discretionary{\hbox{\char`\,}}{\Wrappedafterbreak}{\hbox{\char`\,}}}%
            \lccode`\~`\;\lowercase{\def~}{\discretionary{\hbox{\char`\;}}{\Wrappedafterbreak}{\hbox{\char`\;}}}%
            \lccode`\~`\:\lowercase{\def~}{\discretionary{\hbox{\char`\:}}{\Wrappedafterbreak}{\hbox{\char`\:}}}%
            \lccode`\~`\?\lowercase{\def~}{\discretionary{\hbox{\char`\?}}{\Wrappedafterbreak}{\hbox{\char`\?}}}%
            \lccode`\~`\!\lowercase{\def~}{\discretionary{\hbox{\char`\!}}{\Wrappedafterbreak}{\hbox{\char`\!}}}%
            \lccode`\~`\/\lowercase{\def~}{\discretionary{\hbox{\char`\/}}{\Wrappedafterbreak}{\hbox{\char`\/}}}%
            \catcode`\.\active
            \catcode`\,\active
            \catcode`\;\active
            \catcode`\:\active
            \catcode`\?\active
            \catcode`\!\active
            \catcode`\/\active
            \lccode`\~`\~
        }
    \makeatother

    \let\OriginalVerbatim=\Verbatim
    \makeatletter
    \renewcommand{\Verbatim}[1][1]{%
        %\parskip\z@skip
        \sbox\Wrappedcontinuationbox {\Wrappedcontinuationsymbol}%
        \sbox\Wrappedvisiblespacebox {\FV@SetupFont\Wrappedvisiblespace}%
        \def\FancyVerbFormatLine ##1{\hsize\linewidth
            \vtop{\raggedright\hyphenpenalty\z@\exhyphenpenalty\z@
                \doublehyphendemerits\z@\finalhyphendemerits\z@
                \strut ##1\strut}%
        }%
        % If the linebreak is at a space, the latter will be displayed as visible
        % space at end of first line, and a continuation symbol starts next line.
        % Stretch/shrink are however usually zero for typewriter font.
        \def\FV@Space {%
            \nobreak\hskip\z@ plus\fontdimen3\font minus\fontdimen4\font
            \discretionary{\copy\Wrappedvisiblespacebox}{\Wrappedafterbreak}
            {\kern\fontdimen2\font}%
        }%

        % Allow breaks at special characters using \PYG... macros.
        \Wrappedbreaksatspecials
        % Breaks at punctuation characters . , ; ? ! and / need catcode=\active
        \OriginalVerbatim[#1,codes*=\Wrappedbreaksatpunct]%
    }
    \makeatother

    % Exact colors from NB
    \definecolor{incolor}{HTML}{303F9F}
    \definecolor{outcolor}{HTML}{D84315}
    \definecolor{cellborder}{HTML}{CFCFCF}
    \definecolor{cellbackground}{HTML}{F7F7F7}

    % prompt
    \makeatletter
    \newcommand{\boxspacing}{\kern\kvtcb@left@rule\kern\kvtcb@boxsep}
    \makeatother
    \newcommand{\prompt}[4]{
        {\ttfamily\llap{{\color{#2}[#3]:\hspace{3pt}#4}}\vspace{-\baselineskip}}
    }
    

    
    % Prevent overflowing lines due to hard-to-break entities
    \sloppy
    % Setup hyperref package
    \hypersetup{
      breaklinks=true,  % so long urls are correctly broken across lines
      colorlinks=true,
      urlcolor=urlcolor,
      linkcolor=linkcolor,
      citecolor=citecolor,
      }
    % Slightly bigger margins than the latex defaults
    
    \geometry{verbose,tmargin=1in,bmargin=1in,lmargin=1in,rmargin=1in}
    
    

\begin{document}
    
    \maketitle
 

    \hypertarget{ux5b9eux9a8cux4e00ux57faux4e8e-dctdft-ux53d8ux6362ux7684ux56feux50cfux6c34ux5370ux7b97ux6cd5}{%
\section{实验一:基于 DCT/DFT
变换的图像水印算法}\label{ux5b9eux9a8cux4e00ux57faux4e8e-dctdft-ux53d8ux6362ux7684ux56feux50cfux6c34ux5370ux7b97ux6cd5}}

    \textbf{目标}:将伪随机序列作为水印序列,嵌入到图像的 DCT 或 DFT
域系数中。通过相关检测判别水印是否存在。

    \textbf{实验原理}:图像进行 DCT 或 DFT 变换(不分块),选择 \(K\)
个最大的系数(也可以尝试取值居中的 \(K\) 个,随机 \(K\)
个等方式)。产生包含 \(K\)
个元素的伪随机数序列,比如符合正态分布。按照以下公式嵌入水印信息:

    \[v_i'=v_i(1+\alpha x_i)\]

    其中,\(x_i\) 表示水印信息,\(v_i\) 表示载体图像系数。

    

    \textbf{代码实现}:

    \begin{enumerate}
\def\labelenumi{\arabic{enumi}.}
\tightlist
\item[(1)]
  基于 DCT 变换
\end{enumerate}

    \begin{tcolorbox}[breakable, size=fbox, boxrule=1pt, pad at break*=1mm,colback=cellbackground, colframe=cellborder]
\prompt{In}{incolor}{1}{\boxspacing}
\begin{Verbatim}[commandchars=\\\{\}]
\PY{k+kn}{import} \PY{n+nn}{numpy} \PY{k}{as} \PY{n+nn}{np}
\PY{k+kn}{import} \PY{n+nn}{cv2}
\end{Verbatim}
\end{tcolorbox}

    \begin{tcolorbox}[breakable, size=fbox, boxrule=1pt, pad at break*=1mm,colback=cellbackground, colframe=cellborder]
\prompt{In}{incolor}{2}{\boxspacing}
\begin{Verbatim}[commandchars=\\\{\}]
\PY{k}{def} \PY{n+nf}{embed\PYZus{}watermark}\PY{p}{(}\PY{n}{image\PYZus{}path}\PY{p}{,} \PY{n}{k}\PY{p}{,} \PY{n}{a}\PY{p}{)}\PY{p}{:}
    \PY{c+c1}{\PYZsh{} 加载图像}
    \PY{n}{image} \PY{o}{=} \PY{n}{cv2}\PY{o}{.}\PY{n}{imread}\PY{p}{(}\PY{n}{image\PYZus{}path}\PY{p}{,} \PY{n}{cv2}\PY{o}{.}\PY{n}{IMREAD\PYZus{}GRAYSCALE}\PY{p}{)}

    \PY{c+c1}{\PYZsh{} 执行DCT或DFT变换}
    \PY{n}{transformed} \PY{o}{=} \PY{n}{cv2}\PY{o}{.}\PY{n}{dct}\PY{p}{(}\PY{n}{np}\PY{o}{.}\PY{n}{float32}\PY{p}{(}\PY{n}{image}\PY{p}{)}\PY{p}{)}  \PY{c+c1}{\PYZsh{} 使用DCT变换,或者可以使用cv2.dft()函数进行DFT变换}

    \PY{c+c1}{\PYZsh{} 将变换系数展平}
    \PY{n}{flat\PYZus{}transformed} \PY{o}{=} \PY{n}{transformed}\PY{o}{.}\PY{n}{ravel}\PY{p}{(}\PY{p}{)}

    \PY{c+c1}{\PYZsh{} 获取最大系数的索引}
    \PY{n}{max\PYZus{}indices} \PY{o}{=} \PY{n}{np}\PY{o}{.}\PY{n}{unravel\PYZus{}index}\PY{p}{(}\PY{n}{np}\PY{o}{.}\PY{n}{argsort}\PY{p}{(}\PY{n}{np}\PY{o}{.}\PY{n}{abs}\PY{p}{(}\PY{n}{flat\PYZus{}transformed}\PY{p}{)}\PY{p}{)}\PY{p}{,} \PY{n}{transformed}\PY{o}{.}\PY{n}{shape}\PY{p}{)}

    \PY{c+c1}{\PYZsh{} 选择前k个最大的系数位置}
    \PY{n}{selected\PYZus{}indices} \PY{o}{=} \PY{n}{np}\PY{o}{.}\PY{n}{column\PYZus{}stack}\PY{p}{(}\PY{n}{max\PYZus{}indices}\PY{p}{)}\PY{p}{[}\PY{o}{\PYZhy{}}\PY{n}{k}\PY{p}{:}\PY{p}{]}

    \PY{c+c1}{\PYZsh{} 产生伪随机数序列}
    \PY{n}{watermark\PYZus{}info} \PY{o}{=} \PY{n}{np}\PY{o}{.}\PY{n}{random}\PY{o}{.}\PY{n}{normal}\PY{p}{(}\PY{n}{size}\PY{o}{=}\PY{n}{k}\PY{p}{)}

    \PY{c+c1}{\PYZsh{} 嵌入水印信息}
    \PY{n}{embedded\PYZus{}transformed} \PY{o}{=} \PY{n}{transformed}\PY{o}{.}\PY{n}{copy}\PY{p}{(}\PY{p}{)}
    \PY{k}{for} \PY{n}{index}\PY{p}{,} \PY{n}{value} \PY{o+ow}{in} \PY{n+nb}{zip}\PY{p}{(}\PY{n}{selected\PYZus{}indices}\PY{p}{,} \PY{n}{watermark\PYZus{}info}\PY{p}{)}\PY{p}{:}
        \PY{n}{embedded\PYZus{}transformed}\PY{p}{[}\PY{n}{index}\PY{p}{[}\PY{l+m+mi}{0}\PY{p}{]}\PY{p}{]}\PY{p}{[}\PY{n}{index}\PY{p}{[}\PY{l+m+mi}{1}\PY{p}{]}\PY{p}{]} \PY{o}{*}\PY{o}{=} \PY{p}{(}\PY{l+m+mi}{1} \PY{o}{+} \PY{n}{a} \PY{o}{*} \PY{n}{value}\PY{p}{)}

    \PY{c+c1}{\PYZsh{} 执行逆DCT或逆DFT变换}
    \PY{n}{embedded\PYZus{}image} \PY{o}{=} \PY{n}{cv2}\PY{o}{.}\PY{n}{idct}\PY{p}{(}\PY{n}{embedded\PYZus{}transformed}\PY{p}{)}  \PY{c+c1}{\PYZsh{} 使用逆DCT变换,或者可以使用cv2.idft()函数进行逆DFT变换}

    \PY{k}{return} \PY{n}{embedded\PYZus{}image}\PY{p}{,} \PY{n}{watermark\PYZus{}info}\PY{p}{,} \PY{n}{selected\PYZus{}indices}
\end{Verbatim}
\end{tcolorbox}

    \begin{tcolorbox}[breakable, size=fbox, boxrule=1pt, pad at break*=1mm,colback=cellbackground, colframe=cellborder]
\prompt{In}{incolor}{3}{\boxspacing}
\begin{Verbatim}[commandchars=\\\{\}]
\PY{k}{def} \PY{n+nf}{detect\PYZus{}watermark}\PY{p}{(}\PY{n}{embedded\PYZus{}image}\PY{p}{,} \PY{n}{original\PYZus{}image}\PY{p}{,} \PY{n}{selected\PYZus{}indices}\PY{p}{,} \PY{n}{a}\PY{p}{)}\PY{p}{:}
    \PY{c+c1}{\PYZsh{} 执行DCT或DFT变换}
    \PY{n}{embedded\PYZus{}transformed} \PY{o}{=} \PY{n}{cv2}\PY{o}{.}\PY{n}{dct}\PY{p}{(}\PY{n}{np}\PY{o}{.}\PY{n}{float32}\PY{p}{(}\PY{n}{embedded\PYZus{}image}\PY{p}{)}\PY{p}{)}  \PY{c+c1}{\PYZsh{} 使用DCT变换,或者可以使用cv2.dft()函数进行DFT变换}
    \PY{n}{original\PYZus{}transformed} \PY{o}{=} \PY{n}{cv2}\PY{o}{.}\PY{n}{dct}\PY{p}{(}\PY{n}{np}\PY{o}{.}\PY{n}{float32}\PY{p}{(}\PY{n}{original\PYZus{}image}\PY{p}{)}\PY{p}{)}

    \PY{c+c1}{\PYZsh{} 提取嵌入的水印信息}
    \PY{n}{extracted\PYZus{}watermark} \PY{o}{=} \PY{p}{[}\PY{p}{]}
    \PY{k}{for} \PY{n}{\PYZus{}}\PY{p}{,} \PY{n}{index} \PY{o+ow}{in} \PY{n+nb}{enumerate}\PY{p}{(}\PY{n}{selected\PYZus{}indices}\PY{p}{)}\PY{p}{:}
        \PY{n}{extracted\PYZus{}watermark}\PY{o}{.}\PY{n}{append}\PY{p}{(}
            \PY{p}{(}\PY{n}{embedded\PYZus{}transformed}\PY{p}{[}\PY{n}{index}\PY{p}{[}\PY{l+m+mi}{0}\PY{p}{]}\PY{p}{]}\PY{p}{[}\PY{n}{index}\PY{p}{[}\PY{l+m+mi}{1}\PY{p}{]}\PY{p}{]} \PY{o}{\PYZhy{}} \PY{n}{original\PYZus{}transformed}\PY{p}{[}\PY{n}{index}\PY{p}{[}\PY{l+m+mi}{0}\PY{p}{]}\PY{p}{]}\PY{p}{[}\PY{n}{index}\PY{p}{[}\PY{l+m+mi}{1}\PY{p}{]}\PY{p}{]}\PY{p}{)} \PY{o}{/} \PY{p}{(}\PY{n}{a} \PY{o}{*} \PY{n}{original\PYZus{}transformed}\PY{p}{[}\PY{n}{index}\PY{p}{[}\PY{l+m+mi}{0}\PY{p}{]}\PY{p}{]}\PY{p}{[}\PY{n}{index}\PY{p}{[}\PY{l+m+mi}{1}\PY{p}{]}\PY{p}{]}\PY{p}{)}\PY{p}{)}

    \PY{k}{return} \PY{n}{extracted\PYZus{}watermark}
\end{Verbatim}
\end{tcolorbox}

    \begin{tcolorbox}[breakable, size=fbox, boxrule=1pt, pad at break*=1mm,colback=cellbackground, colframe=cellborder]
\prompt{In}{incolor}{4}{\boxspacing}
\begin{Verbatim}[commandchars=\\\{\}]
\PY{k}{def} \PY{n+nf}{calculate\PYZus{}psnr}\PY{p}{(}\PY{n}{image1}\PY{p}{,} \PY{n}{image2}\PY{p}{)}\PY{p}{:}
    \PY{k}{assert} \PY{n}{image1}\PY{o}{.}\PY{n}{shape} \PY{o}{==} \PY{n}{image2}\PY{o}{.}\PY{n}{shape}\PY{p}{,} \PY{p}{(}\PY{l+s+s1}{\PYZsq{}}\PY{l+s+s1}{错误:两个输⼊图像的⼤⼩不⼀致: }\PY{l+s+si}{\PYZob{}image1.shape\PYZcb{}}\PY{l+s+s1}{, }\PY{l+s+si}{\PYZob{}image2.shape\PYZcb{}}\PY{l+s+s1}{.}\PY{l+s+s1}{\PYZsq{}}\PY{p}{)}
    \PY{n}{mse} \PY{o}{=} \PY{n}{np}\PY{o}{.}\PY{n}{mean}\PY{p}{(}\PY{p}{(}\PY{n}{image1} \PY{o}{\PYZhy{}} \PY{n}{image2}\PY{p}{)} \PY{o}{*}\PY{o}{*} \PY{l+m+mi}{2}\PY{p}{)}
    \PY{k}{if} \PY{n}{mse} \PY{o}{==} \PY{l+m+mi}{0}\PY{p}{:}
        \PY{n+nb}{print}\PY{p}{(}\PY{l+s+s1}{\PYZsq{}}\PY{l+s+s1}{两幅图像完全⼀样}\PY{l+s+s1}{\PYZsq{}}\PY{p}{)}
        \PY{k}{return} \PY{l+m+mi}{200}
    \PY{n}{psnr} \PY{o}{=} \PY{l+m+mi}{10} \PY{o}{*} \PY{n}{np}\PY{o}{.}\PY{n}{log10}\PY{p}{(}\PY{p}{(}\PY{l+m+mi}{255} \PY{o}{*}\PY{o}{*} \PY{l+m+mi}{2}\PY{p}{)} \PY{o}{/} \PY{n}{mse}\PY{p}{)}
    \PY{k}{return} \PY{n}{psnr}
\end{Verbatim}
\end{tcolorbox}

    \begin{tcolorbox}[breakable, size=fbox, boxrule=1pt, pad at break*=1mm,colback=cellbackground, colframe=cellborder]
\prompt{In}{incolor}{5}{\boxspacing}
\begin{Verbatim}[commandchars=\\\{\}]
\PY{n}{image\PYZus{}path} \PY{o}{=} \PY{l+s+s1}{\PYZsq{}}\PY{l+s+s1}{Lena\PYZus{}256.bmp}\PY{l+s+s1}{\PYZsq{}}
\PY{n}{k} \PY{o}{=} \PY{l+m+mi}{10}  \PY{c+c1}{\PYZsh{} 选择的DCT或DFT系数个数}
\PY{n}{a} \PY{o}{=} \PY{l+m+mf}{0.1}  \PY{c+c1}{\PYZsh{} 嵌入系数}

\PY{c+c1}{\PYZsh{} 嵌入水印}
\PY{n}{embedded\PYZus{}image}\PY{p}{,} \PY{n}{watermark\PYZus{}info}\PY{p}{,} \PY{n}{selected\PYZus{}indices} \PY{o}{=} \PY{n}{embed\PYZus{}watermark}\PY{p}{(}\PY{n}{image\PYZus{}path}\PY{p}{,} \PY{n}{k}\PY{p}{,} \PY{n}{a}\PY{p}{)}

\PY{c+c1}{\PYZsh{} 计算嵌入水印后图像的PSNR值}
\PY{n}{original\PYZus{}image} \PY{o}{=} \PY{n}{cv2}\PY{o}{.}\PY{n}{imread}\PY{p}{(}\PY{n}{image\PYZus{}path}\PY{p}{,} \PY{n}{cv2}\PY{o}{.}\PY{n}{IMREAD\PYZus{}GRAYSCALE}\PY{p}{)}
\PY{n}{psnr} \PY{o}{=} \PY{n}{calculate\PYZus{}psnr}\PY{p}{(}\PY{n}{original\PYZus{}image}\PY{p}{,} \PY{n}{embedded\PYZus{}image}\PY{p}{)}

\PY{c+c1}{\PYZsh{} 提取水印}
\PY{n}{extracted\PYZus{}watermark} \PY{o}{=} \PY{n}{detect\PYZus{}watermark}\PY{p}{(}\PY{n}{embedded\PYZus{}image}\PY{p}{,} \PY{n}{original\PYZus{}image}\PY{p}{,} \PY{n}{selected\PYZus{}indices}\PY{p}{,} \PY{n}{a}\PY{p}{)}

\PY{c+c1}{\PYZsh{} 计算提取的水印与原始水印的相关值}
\PY{n}{correlation} \PY{o}{=} \PY{n}{np}\PY{o}{.}\PY{n}{corrcoef}\PY{p}{(}\PY{n}{watermark\PYZus{}info}\PY{p}{,} \PY{n}{extracted\PYZus{}watermark}\PY{p}{)}\PY{p}{[}\PY{l+m+mi}{0}\PY{p}{,} \PY{l+m+mi}{1}\PY{p}{]}

\PY{n+nb}{print}\PY{p}{(}\PY{l+s+s2}{\PYZdq{}}\PY{l+s+s2}{PSNR:}\PY{l+s+s2}{\PYZdq{}}\PY{p}{,} \PY{n}{psnr}\PY{p}{)}
\PY{n+nb}{print}\PY{p}{(}\PY{l+s+s2}{\PYZdq{}}\PY{l+s+s2}{Correlation:}\PY{l+s+s2}{\PYZdq{}}\PY{p}{,} \PY{n}{correlation}\PY{p}{)}

\PY{n}{threshold} \PY{o}{=} \PY{l+m+mf}{0.5}
\PY{k}{if} \PY{n}{correlation} \PY{o}{\PYZgt{}} \PY{n}{threshold}\PY{p}{:}
    \PY{n+nb}{print}\PY{p}{(}\PY{l+s+s2}{\PYZdq{}}\PY{l+s+s2}{水印存在}\PY{l+s+s2}{\PYZdq{}}\PY{p}{)}
\PY{k}{else}\PY{p}{:}
    \PY{n+nb}{print}\PY{p}{(}\PY{l+s+s2}{\PYZdq{}}\PY{l+s+s2}{水印不存在}\PY{l+s+s2}{\PYZdq{}}\PY{p}{)}
\end{Verbatim}
\end{tcolorbox}

    \begin{Verbatim}[commandchars=\\\{\}]
PSNR: 38.82497554997011
Correlation: 0.9999999999994027
水印存在
    \end{Verbatim}

    \begin{enumerate}
\def\labelenumi{\arabic{enumi}.}
\setcounter{enumi}{1}
\tightlist
\item[(2)]
  基于小波变换实现
\end{enumerate}

    \begin{tcolorbox}[breakable, size=fbox, boxrule=1pt, pad at break*=1mm,colback=cellbackground, colframe=cellborder]
\prompt{In}{incolor}{6}{\boxspacing}
\begin{Verbatim}[commandchars=\\\{\}]
\PY{k+kn}{import} \PY{n+nn}{pywt}
\PY{k+kn}{import} \PY{n+nn}{numpy} \PY{k}{as} \PY{n+nn}{np}
\PY{k+kn}{import} \PY{n+nn}{cv2}
\end{Verbatim}
\end{tcolorbox}

    \begin{tcolorbox}[breakable, size=fbox, boxrule=1pt, pad at break*=1mm,colback=cellbackground, colframe=cellborder]
\prompt{In}{incolor}{7}{\boxspacing}
\begin{Verbatim}[commandchars=\\\{\}]
\PY{k}{def} \PY{n+nf}{embed\PYZus{}watermark}\PY{p}{(}\PY{n}{image}\PY{p}{,} \PY{n}{watermark}\PY{p}{,} \PY{n}{k}\PY{p}{,} \PY{n}{a}\PY{p}{)}\PY{p}{:}
    \PY{c+c1}{\PYZsh{} 进行小波变换}
    \PY{n}{coeffs} \PY{o}{=} \PY{n}{pywt}\PY{o}{.}\PY{n}{dwt2}\PY{p}{(}\PY{n}{image}\PY{p}{,} \PY{l+s+s1}{\PYZsq{}}\PY{l+s+s1}{haar}\PY{l+s+s1}{\PYZsq{}}\PY{p}{)}
    \PY{n}{LL}\PY{p}{,} \PY{p}{(}\PY{n}{LH}\PY{p}{,} \PY{n}{HL}\PY{p}{,} \PY{n}{HH}\PY{p}{)} \PY{o}{=} \PY{n}{coeffs}

    \PY{c+c1}{\PYZsh{} 由于人眼对高频信息不敏感,将水印信息嵌入到高频小波系数中}
    \PY{n}{watermark\PYZus{}indices} \PY{o}{=} \PY{n}{np}\PY{o}{.}\PY{n}{unravel\PYZus{}index}\PY{p}{(}\PY{n}{np}\PY{o}{.}\PY{n}{argsort}\PY{p}{(}\PY{n}{np}\PY{o}{.}\PY{n}{abs}\PY{p}{(}\PY{n}{HH}\PY{o}{.}\PY{n}{ravel}\PY{p}{(}\PY{p}{)}\PY{p}{)}\PY{p}{,} \PY{n}{axis}\PY{o}{=}\PY{k+kc}{None}\PY{p}{)}\PY{p}{,} \PY{n}{HH}\PY{o}{.}\PY{n}{shape}\PY{p}{)}
    \PY{n}{selected\PYZus{}indices} \PY{o}{=} \PY{n}{np}\PY{o}{.}\PY{n}{column\PYZus{}stack}\PY{p}{(}\PY{n}{watermark\PYZus{}indices}\PY{p}{)}\PY{p}{[}\PY{o}{\PYZhy{}}\PY{n}{k}\PY{p}{:}\PY{p}{]}
    \PY{n}{HH\PYZus{}watermarked} \PY{o}{=} \PY{n}{HH}\PY{o}{.}\PY{n}{copy}\PY{p}{(}\PY{p}{)}
    \PY{k}{for} \PY{n}{indices}\PY{p}{,} \PY{n}{value} \PY{o+ow}{in} \PY{n+nb}{zip}\PY{p}{(}\PY{n}{selected\PYZus{}indices}\PY{p}{,} \PY{n}{watermark}\PY{p}{)}\PY{p}{:}
        \PY{n}{HH\PYZus{}watermarked}\PY{p}{[}\PY{n}{indices}\PY{p}{[}\PY{l+m+mi}{0}\PY{p}{]}\PY{p}{]}\PY{p}{[}\PY{n}{indices}\PY{p}{[}\PY{l+m+mi}{1}\PY{p}{]}\PY{p}{]} \PY{o}{*}\PY{o}{=} \PY{p}{(}\PY{l+m+mi}{1} \PY{o}{+} \PY{n}{a} \PY{o}{*} \PY{n}{value}\PY{p}{)}

    \PY{c+c1}{\PYZsh{} 逆小波变换恢复图像}
    \PY{n}{watermarked\PYZus{}coeffs} \PY{o}{=} \PY{p}{(}\PY{n}{LL}\PY{p}{,} \PY{p}{(}\PY{n}{LH}\PY{p}{,} \PY{n}{HL}\PY{p}{,} \PY{n}{HH\PYZus{}watermarked}\PY{p}{)}\PY{p}{)}
    \PY{n}{watermarked\PYZus{}image} \PY{o}{=} \PY{n}{pywt}\PY{o}{.}\PY{n}{idwt2}\PY{p}{(}\PY{n}{watermarked\PYZus{}coeffs}\PY{p}{,} \PY{l+s+s1}{\PYZsq{}}\PY{l+s+s1}{haar}\PY{l+s+s1}{\PYZsq{}}\PY{p}{)}
    \PY{n}{watermarked\PYZus{}image} \PY{o}{=} \PY{n}{np}\PY{o}{.}\PY{n}{clip}\PY{p}{(}\PY{n}{watermarked\PYZus{}image}\PY{p}{,} \PY{l+m+mi}{0}\PY{p}{,} \PY{l+m+mi}{255}\PY{p}{)}\PY{o}{.}\PY{n}{astype}\PY{p}{(}\PY{n}{np}\PY{o}{.}\PY{n}{uint8}\PY{p}{)}

    \PY{k}{return} \PY{n}{watermarked\PYZus{}image}\PY{p}{,} \PY{n}{selected\PYZus{}indices}
\end{Verbatim}
\end{tcolorbox}

    \begin{tcolorbox}[breakable, size=fbox, boxrule=1pt, pad at break*=1mm,colback=cellbackground, colframe=cellborder]
\prompt{In}{incolor}{8}{\boxspacing}
\begin{Verbatim}[commandchars=\\\{\}]
\PY{k}{def} \PY{n+nf}{extract\PYZus{}watermark}\PY{p}{(}\PY{n}{image}\PY{p}{,} \PY{n}{original\PYZus{}image}\PY{p}{,} \PY{n}{selected\PYZus{}indices}\PY{p}{,} \PY{n}{a}\PY{p}{)}\PY{p}{:}
    \PY{c+c1}{\PYZsh{} 进行小波变换}
    \PY{n}{coeffs} \PY{o}{=} \PY{n}{pywt}\PY{o}{.}\PY{n}{dwt2}\PY{p}{(}\PY{n}{image}\PY{p}{,} \PY{l+s+s1}{\PYZsq{}}\PY{l+s+s1}{haar}\PY{l+s+s1}{\PYZsq{}}\PY{p}{)}
    \PY{n}{LL}\PY{p}{,} \PY{p}{(}\PY{n}{LH}\PY{p}{,} \PY{n}{HL}\PY{p}{,} \PY{n}{HH}\PY{p}{)} \PY{o}{=} \PY{n}{coeffs}

    \PY{n}{original\PYZus{}coeffs} \PY{o}{=} \PY{n}{pywt}\PY{o}{.}\PY{n}{dwt2}\PY{p}{(}\PY{n}{original\PYZus{}image}\PY{p}{,} \PY{l+s+s1}{\PYZsq{}}\PY{l+s+s1}{haar}\PY{l+s+s1}{\PYZsq{}}\PY{p}{)}
    \PY{n}{original\PYZus{}LL}\PY{p}{,} \PY{p}{(}\PY{n}{original\PYZus{}LH}\PY{p}{,} \PY{n}{original\PYZus{}HL}\PY{p}{,} \PY{n}{original\PYZus{}HH}\PY{p}{)} \PY{o}{=} \PY{n}{original\PYZus{}coeffs}

    \PY{c+c1}{\PYZsh{} 提取嵌入的水印信息}
    \PY{n}{extracted\PYZus{}watermark} \PY{o}{=} \PY{p}{[}\PY{p}{]}
    \PY{k}{for} \PY{n}{indices} \PY{o+ow}{in} \PY{n}{selected\PYZus{}indices}\PY{p}{:}
        \PY{n}{extracted\PYZus{}watermark}\PY{o}{.}\PY{n}{append}\PY{p}{(}\PY{p}{(}\PY{n}{HH}\PY{p}{[}\PY{n}{indices}\PY{p}{[}\PY{l+m+mi}{0}\PY{p}{]}\PY{p}{]}\PY{p}{[}\PY{n}{indices}\PY{p}{[}\PY{l+m+mi}{1}\PY{p}{]}\PY{p}{]} \PY{o}{\PYZhy{}} \PY{n}{original\PYZus{}HH}\PY{p}{[}\PY{n}{indices}\PY{p}{[}\PY{l+m+mi}{0}\PY{p}{]}\PY{p}{]}\PY{p}{[}\PY{n}{indices}\PY{p}{[}\PY{l+m+mi}{1}\PY{p}{]}\PY{p}{]}\PY{p}{)} \PY{o}{/} \PY{p}{(}\PY{n}{a} \PY{o}{*} \PY{n}{original\PYZus{}HH}\PY{p}{[}\PY{n}{indices}\PY{p}{[}\PY{l+m+mi}{0}\PY{p}{]}\PY{p}{]}\PY{p}{[}\PY{n}{indices}\PY{p}{[}\PY{l+m+mi}{1}\PY{p}{]}\PY{p}{]}\PY{p}{)}\PY{p}{)}

    \PY{k}{return} \PY{n}{extracted\PYZus{}watermark}
\end{Verbatim}
\end{tcolorbox}

    \begin{tcolorbox}[breakable, size=fbox, boxrule=1pt, pad at break*=1mm,colback=cellbackground, colframe=cellborder]
\prompt{In}{incolor}{9}{\boxspacing}
\begin{Verbatim}[commandchars=\\\{\}]
\PY{c+c1}{\PYZsh{} 加载图像}
\PY{n}{image} \PY{o}{=} \PY{n}{cv2}\PY{o}{.}\PY{n}{imread}\PY{p}{(}\PY{l+s+s1}{\PYZsq{}}\PY{l+s+s1}{./Lena\PYZus{}256.bmp}\PY{l+s+s1}{\PYZsq{}}\PY{p}{,} \PY{l+m+mi}{0}\PY{p}{)}

\PY{c+c1}{\PYZsh{} 生成水印}
\PY{n}{k} \PY{o}{=} \PY{l+m+mi}{10}
\PY{n}{watermark} \PY{o}{=} \PY{n}{np}\PY{o}{.}\PY{n}{random}\PY{o}{.}\PY{n}{normal}\PY{p}{(}\PY{n}{size}\PY{o}{=}\PY{n}{k}\PY{p}{)}
\PY{n}{a} \PY{o}{=} \PY{l+m+mf}{0.1}

\PY{c+c1}{\PYZsh{} 嵌入水印}
\PY{n}{watermarked\PYZus{}image}\PY{p}{,} \PY{n}{selected\PYZus{}indices} \PY{o}{=} \PY{n}{embed\PYZus{}watermark}\PY{p}{(}\PY{n}{image}\PY{p}{,} \PY{n}{watermark}\PY{p}{,} \PY{n}{k}\PY{p}{,} \PY{n}{a}\PY{p}{)}

\PY{c+c1}{\PYZsh{} 提取水印}
\PY{n}{extracted\PYZus{}watermark} \PY{o}{=} \PY{n}{extract\PYZus{}watermark}\PY{p}{(}\PY{n}{watermarked\PYZus{}image}\PY{p}{,} \PY{n}{image}\PY{p}{,} \PY{n}{selected\PYZus{}indices}\PY{p}{,} \PY{n}{a}\PY{p}{)}


\PY{k}{def} \PY{n+nf}{calculate\PYZus{}psnr}\PY{p}{(}\PY{n}{image1}\PY{p}{,} \PY{n}{image2}\PY{p}{)}\PY{p}{:}
    \PY{k}{assert} \PY{n}{image1}\PY{o}{.}\PY{n}{shape} \PY{o}{==} \PY{n}{image2}\PY{o}{.}\PY{n}{shape}\PY{p}{,} \PY{p}{(}\PY{l+s+s1}{\PYZsq{}}\PY{l+s+s1}{错误:两个输⼊图像的⼤⼩不⼀致: }\PY{l+s+si}{\PYZob{}image1.shape\PYZcb{}}\PY{l+s+s1}{, }\PY{l+s+si}{\PYZob{}image2.shape\PYZcb{}}\PY{l+s+s1}{.}\PY{l+s+s1}{\PYZsq{}}\PY{p}{)}
    \PY{n}{mse} \PY{o}{=} \PY{n}{np}\PY{o}{.}\PY{n}{mean}\PY{p}{(}\PY{p}{(}\PY{n}{image1} \PY{o}{\PYZhy{}} \PY{n}{image2}\PY{p}{)} \PY{o}{*}\PY{o}{*} \PY{l+m+mi}{2}\PY{p}{)}
    \PY{k}{if} \PY{n}{mse} \PY{o}{==} \PY{l+m+mi}{0}\PY{p}{:}
        \PY{n+nb}{print}\PY{p}{(}\PY{l+s+s1}{\PYZsq{}}\PY{l+s+s1}{两幅图像完全⼀样}\PY{l+s+s1}{\PYZsq{}}\PY{p}{)}
        \PY{k}{return} \PY{l+m+mi}{200}
    \PY{n}{psnr} \PY{o}{=} \PY{l+m+mi}{10} \PY{o}{*} \PY{n}{np}\PY{o}{.}\PY{n}{log10}\PY{p}{(}\PY{p}{(}\PY{l+m+mi}{255} \PY{o}{*}\PY{o}{*} \PY{l+m+mi}{2}\PY{p}{)} \PY{o}{/} \PY{n}{mse}\PY{p}{)}
    \PY{k}{return} \PY{n}{psnr}


\PY{n}{psnr} \PY{o}{=} \PY{n}{calculate\PYZus{}psnr}\PY{p}{(}\PY{n}{image}\PY{p}{,} \PY{n}{watermarked\PYZus{}image}\PY{p}{)}
\PY{n+nb}{print}\PY{p}{(}\PY{l+s+s2}{\PYZdq{}}\PY{l+s+s2}{PSNR:}\PY{l+s+s2}{\PYZdq{}}\PY{p}{,} \PY{n}{psnr}\PY{p}{)}

\PY{c+c1}{\PYZsh{} 判断水印是否存在}
\PY{n}{correlation} \PY{o}{=} \PY{n}{np}\PY{o}{.}\PY{n}{corrcoef}\PY{p}{(}\PY{n}{watermark}\PY{p}{,} \PY{n}{extracted\PYZus{}watermark}\PY{p}{)}\PY{p}{[}\PY{l+m+mi}{0}\PY{p}{,} \PY{l+m+mi}{1}\PY{p}{]}
\PY{n+nb}{print}\PY{p}{(}\PY{l+s+s2}{\PYZdq{}}\PY{l+s+s2}{Correlation:}\PY{l+s+s2}{\PYZdq{}}\PY{p}{,} \PY{n}{correlation}\PY{p}{)}

\PY{n}{threshold} \PY{o}{=} \PY{l+m+mf}{0.5}
\PY{k}{if} \PY{n}{correlation} \PY{o}{\PYZgt{}} \PY{n}{threshold}\PY{p}{:}
    \PY{n+nb}{print}\PY{p}{(}\PY{l+s+s2}{\PYZdq{}}\PY{l+s+s2}{水印存在}\PY{l+s+s2}{\PYZdq{}}\PY{p}{)}
\PY{k}{else}\PY{p}{:}
    \PY{n+nb}{print}\PY{p}{(}\PY{l+s+s2}{\PYZdq{}}\PY{l+s+s2}{水印不存在}\PY{l+s+s2}{\PYZdq{}}\PY{p}{)}
\end{Verbatim}
\end{tcolorbox}

    \begin{Verbatim}[commandchars=\\\{\}]
PSNR: 66.49556919907863
Correlation: 0.9987793429660823
水印存在
    \end{Verbatim}

    可以看到,无论是基于 DCT
变换还是基于小波变换的实现,都可以较为准确地提取出嵌入的水印,但相较于
DCT 变换,小波变换造成的载体图像失真更小

    \hypertarget{ux5b9eux9a8cux4e8cux5229-dct-ux57dfux56feux50cfux5757ux7cfbux6570ux5173ux7cfbux7684ux5370ux7b97ux6cd5}{%
\section{实验二:利⽤ DCT
域图像块系数关系的⽔印算法}\label{ux5b9eux9a8cux4e8cux5229-dct-ux57dfux56feux50cfux5757ux7cfbux6570ux5173ux7cfbux7684ux5370ux7b97ux6cd5}}

    \textbf{目标}:将二值图像作为水印,嵌入到灰度载体图像中。

    \textbf{实验原理}:图像分为 \(8\times8\) 的图像块,每个图像块嵌入 \(1\)
比特水印。在每个图像块中, 选出 \(2\)
个位置的系数,比较他们的大小。具体的,选出 \(B(u_1, v_1)\)和
\(B(u_2, v_2)\)两个系数,如果嵌入 \(1\),使得
\(B(u_1, v_1)>B(u_2, v_2)\);如果嵌入 0,使得
\(B(u_1, v_1)<B(u_2, v_2)\)。 最后,调整两个系数间的差距,保证鲁棒性。

    \begin{tcolorbox}[breakable, size=fbox, boxrule=1pt, pad at break*=1mm,colback=cellbackground, colframe=cellborder]
\prompt{In}{incolor}{10}{\boxspacing}
\begin{Verbatim}[commandchars=\\\{\}]
\PY{k+kn}{import} \PY{n+nn}{cv2}
\PY{k+kn}{import} \PY{n+nn}{numpy} \PY{k}{as} \PY{n+nn}{np}
\PY{k+kn}{import} \PY{n+nn}{matplotlib}\PY{n+nn}{.}\PY{n+nn}{pyplot} \PY{k}{as} \PY{n+nn}{plt}
\end{Verbatim}
\end{tcolorbox}

    \begin{tcolorbox}[breakable, size=fbox, boxrule=1pt, pad at break*=1mm,colback=cellbackground, colframe=cellborder]
\prompt{In}{incolor}{11}{\boxspacing}
\begin{Verbatim}[commandchars=\\\{\}]
\PY{k}{class} \PY{n+nc}{DCT\PYZus{}Embed}\PY{p}{(}\PY{n+nb}{object}\PY{p}{)}\PY{p}{:}
    \PY{k}{def} \PY{n+nf+fm}{\PYZus{}\PYZus{}init\PYZus{}\PYZus{}}\PY{p}{(}\PY{n+nb+bp}{self}\PY{p}{,} \PY{n}{background}\PY{p}{,} \PY{n}{watermark}\PY{p}{,} \PY{n}{block\PYZus{}size}\PY{o}{=}\PY{l+m+mi}{8}\PY{p}{,} \PY{n}{alpha}\PY{o}{=}\PY{l+m+mi}{30}\PY{p}{)}\PY{p}{:}
        \PY{n}{b\PYZus{}h}\PY{p}{,} \PY{n}{b\PYZus{}w} \PY{o}{=} \PY{n}{background}\PY{o}{.}\PY{n}{shape}\PY{p}{[}\PY{p}{:}\PY{l+m+mi}{2}\PY{p}{]}
        \PY{n}{w\PYZus{}h}\PY{p}{,} \PY{n}{w\PYZus{}w} \PY{o}{=} \PY{n}{watermark}\PY{o}{.}\PY{n}{shape}\PY{p}{[}\PY{p}{:}\PY{l+m+mi}{2}\PY{p}{]}
        \PY{k}{assert} \PY{n}{w\PYZus{}h} \PY{o}{\PYZlt{}}\PY{o}{=} \PY{n}{b\PYZus{}h} \PY{o}{/} \PY{n}{block\PYZus{}size} \PY{o+ow}{and} \PY{n}{w\PYZus{}w} \PY{o}{\PYZlt{}}\PY{o}{=} \PY{n}{b\PYZus{}w} \PY{o}{/} \PY{n}{block\PYZus{}size}\PY{p}{,} \PYZbs{}
            \PY{l+s+s2}{\PYZdq{}}\PY{l+s+se}{\PYZbs{}r}\PY{l+s+se}{\PYZbs{}n}\PY{l+s+s2}{请确保您的的水印图像尺寸 不大于 背景图像尺寸的1/}\PY{l+s+si}{\PYZob{}:\PYZcb{}}\PY{l+s+se}{\PYZbs{}r}\PY{l+s+se}{\PYZbs{}n}\PY{l+s+s2}{background尺寸}\PY{l+s+si}{\PYZob{}:\PYZcb{}}\PY{l+s+se}{\PYZbs{}r}\PY{l+s+se}{\PYZbs{}n}\PY{l+s+s2}{watermark尺寸}\PY{l+s+si}{\PYZob{}:\PYZcb{}}\PY{l+s+s2}{\PYZdq{}}\PY{o}{.}\PY{n}{format}\PY{p}{(}
                \PY{n}{block\PYZus{}size}\PY{p}{,} \PY{n}{background}\PY{o}{.}\PY{n}{shape}\PY{p}{,} \PY{n}{watermark}\PY{o}{.}\PY{n}{shape}
            \PY{p}{)}

        \PY{c+c1}{\PYZsh{} 保存参数}
        \PY{n+nb+bp}{self}\PY{o}{.}\PY{n}{block\PYZus{}size} \PY{o}{=} \PY{n}{block\PYZus{}size}
        \PY{c+c1}{\PYZsh{} 水印强度控制}
        \PY{n+nb+bp}{self}\PY{o}{.}\PY{n}{alpha} \PY{o}{=} \PY{n}{alpha}

    \PY{k}{def} \PY{n+nf}{dct\PYZus{}blkproc}\PY{p}{(}\PY{n+nb+bp}{self}\PY{p}{,} \PY{n}{background}\PY{p}{)}\PY{p}{:}
\PY{+w}{        }\PY{l+s+sd}{\PYZdq{}\PYZdq{}\PYZdq{}}
\PY{l+s+sd}{        对background进行分块,然后进行dct变换,得到dct变换后的矩阵}

\PY{l+s+sd}{        :param image: 输入图像}
\PY{l+s+sd}{        :param split\PYZus{}w: 分割的每个patch的w}
\PY{l+s+sd}{        :param split\PYZus{}h: 分割的每个patch的h}
\PY{l+s+sd}{        :return: 经dct变换的分块矩阵、原始的分块矩阵}
\PY{l+s+sd}{        \PYZdq{}\PYZdq{}\PYZdq{}}
        \PY{n}{background\PYZus{}dct\PYZus{}blocks\PYZus{}h} \PY{o}{=} \PY{n}{background}\PY{o}{.}\PY{n}{shape}\PY{p}{[}\PY{l+m+mi}{0}\PY{p}{]} \PY{o}{/}\PY{o}{/} \PY{n+nb+bp}{self}\PY{o}{.}\PY{n}{block\PYZus{}size}  \PY{c+c1}{\PYZsh{} 高度}
        \PY{n}{background\PYZus{}dct\PYZus{}blocks\PYZus{}w} \PY{o}{=} \PY{n}{background}\PY{o}{.}\PY{n}{shape}\PY{p}{[}\PY{l+m+mi}{1}\PY{p}{]} \PY{o}{/}\PY{o}{/} \PY{n+nb+bp}{self}\PY{o}{.}\PY{n}{block\PYZus{}size}  \PY{c+c1}{\PYZsh{} 宽度}
        \PY{n}{background\PYZus{}dct\PYZus{}blocks} \PY{o}{=} \PY{n}{np}\PY{o}{.}\PY{n}{zeros}\PY{p}{(}\PY{n}{shape}\PY{o}{=}\PY{p}{(}
            \PY{p}{(}\PY{n}{background\PYZus{}dct\PYZus{}blocks\PYZus{}h}\PY{p}{,} \PY{n}{background\PYZus{}dct\PYZus{}blocks\PYZus{}w}\PY{p}{,} \PY{n+nb+bp}{self}\PY{o}{.}\PY{n}{block\PYZus{}size}\PY{p}{,} \PY{n+nb+bp}{self}\PY{o}{.}\PY{n}{block\PYZus{}size}\PY{p}{)}
        \PY{p}{)}\PY{p}{)}  \PY{c+c1}{\PYZsh{} 前2个维度用来遍历所有block,后2个维度用来存储每个block的DCT变换的值}

        \PY{n}{h\PYZus{}data} \PY{o}{=} \PY{n}{np}\PY{o}{.}\PY{n}{vsplit}\PY{p}{(}\PY{n}{background}\PY{p}{,} \PY{n}{background\PYZus{}dct\PYZus{}blocks\PYZus{}h}\PY{p}{)}  \PY{c+c1}{\PYZsh{} 垂直方向分成background\PYZus{}dct\PYZus{}blocks\PYZus{}h个块}
        \PY{k}{for} \PY{n}{h} \PY{o+ow}{in} \PY{n+nb}{range}\PY{p}{(}\PY{n}{background\PYZus{}dct\PYZus{}blocks\PYZus{}h}\PY{p}{)}\PY{p}{:}
            \PY{n}{block\PYZus{}data} \PY{o}{=} \PY{n}{np}\PY{o}{.}\PY{n}{hsplit}\PY{p}{(}\PY{n}{h\PYZus{}data}\PY{p}{[}\PY{n}{h}\PY{p}{]}\PY{p}{,} \PY{n}{background\PYZus{}dct\PYZus{}blocks\PYZus{}w}\PY{p}{)}  \PY{c+c1}{\PYZsh{} 水平方向分成background\PYZus{}dct\PYZus{}blocks\PYZus{}w个块}
            \PY{k}{for} \PY{n}{w} \PY{o+ow}{in} \PY{n+nb}{range}\PY{p}{(}\PY{n}{background\PYZus{}dct\PYZus{}blocks\PYZus{}w}\PY{p}{)}\PY{p}{:}
                \PY{n}{a\PYZus{}block} \PY{o}{=} \PY{n}{block\PYZus{}data}\PY{p}{[}\PY{n}{w}\PY{p}{]}
                \PY{n}{background\PYZus{}dct\PYZus{}blocks}\PY{p}{[}\PY{n}{h}\PY{p}{,} \PY{n}{w}\PY{p}{,} \PY{o}{.}\PY{o}{.}\PY{o}{.}\PY{p}{]} \PY{o}{=} \PY{n}{cv2}\PY{o}{.}\PY{n}{dct}\PY{p}{(}\PY{n}{a\PYZus{}block}\PY{o}{.}\PY{n}{astype}\PY{p}{(}\PY{n}{np}\PY{o}{.}\PY{n}{float64}\PY{p}{)}\PY{p}{)}  \PY{c+c1}{\PYZsh{} dct变换}
        \PY{k}{return} \PY{n}{background\PYZus{}dct\PYZus{}blocks}

    \PY{k}{def} \PY{n+nf}{dct\PYZus{}embed}\PY{p}{(}\PY{n+nb+bp}{self}\PY{p}{,} \PY{n}{dct\PYZus{}data}\PY{p}{,} \PY{n}{watermark}\PY{p}{)}\PY{p}{:}
\PY{+w}{        }\PY{l+s+sd}{\PYZdq{}\PYZdq{}\PYZdq{}}
\PY{l+s+sd}{        将水印嵌入到载体的dct系数中}
\PY{l+s+sd}{        :param dct\PYZus{}data: 背景图像(载体)的DCT系数}
\PY{l+s+sd}{        :param watermark: 归一化二值图像0\PYZhy{}1 (uint8类型)}
\PY{l+s+sd}{        :return: 空域图像}
\PY{l+s+sd}{        \PYZdq{}\PYZdq{}\PYZdq{}}
        \PY{n}{temp} \PY{o}{=} \PY{n}{watermark}\PY{o}{.}\PY{n}{flatten}\PY{p}{(}\PY{p}{)}
        \PY{k}{assert} \PY{n}{temp}\PY{o}{.}\PY{n}{max}\PY{p}{(}\PY{p}{)} \PY{o}{==} \PY{l+m+mi}{1} \PY{o+ow}{and} \PY{n}{temp}\PY{o}{.}\PY{n}{min}\PY{p}{(}\PY{p}{)} \PY{o}{==} \PY{l+m+mi}{0}\PY{p}{,} \PY{l+s+s2}{\PYZdq{}}\PY{l+s+s2}{为方便处理,请保证输入的watermark是被二值归一化的}\PY{l+s+s2}{\PYZdq{}}

        \PY{k}{for} \PY{n}{h} \PY{o+ow}{in} \PY{n+nb}{range}\PY{p}{(}\PY{n}{watermark}\PY{o}{.}\PY{n}{shape}\PY{p}{[}\PY{l+m+mi}{0}\PY{p}{]}\PY{p}{)}\PY{p}{:}
            \PY{k}{for} \PY{n}{w} \PY{o+ow}{in} \PY{n+nb}{range}\PY{p}{(}\PY{n}{watermark}\PY{o}{.}\PY{n}{shape}\PY{p}{[}\PY{l+m+mi}{1}\PY{p}{]}\PY{p}{)}\PY{p}{:}
                \PY{k}{if} \PY{n}{watermark}\PY{p}{[}\PY{n}{h}\PY{p}{,} \PY{n}{w}\PY{p}{]} \PY{o}{==} \PY{l+m+mi}{0}\PY{p}{:}
                    \PY{k}{if} \PY{n}{dct\PYZus{}data}\PY{p}{[}\PY{n}{h}\PY{p}{,} \PY{n}{w}\PY{p}{,} \PY{l+m+mi}{4}\PY{p}{,} \PY{l+m+mi}{1}\PY{p}{]} \PY{o}{\PYZlt{}} \PY{n}{dct\PYZus{}data}\PY{p}{[}\PY{n}{h}\PY{p}{,} \PY{n}{w}\PY{p}{,} \PY{l+m+mi}{3}\PY{p}{,} \PY{l+m+mi}{2}\PY{p}{]}\PY{p}{:}
                        \PY{n}{tmp} \PY{o}{=} \PY{n}{dct\PYZus{}data}\PY{p}{[}\PY{n}{h}\PY{p}{,} \PY{n}{w}\PY{p}{,} \PY{l+m+mi}{4}\PY{p}{,} \PY{l+m+mi}{1}\PY{p}{]}
                        \PY{n}{dct\PYZus{}data}\PY{p}{[}\PY{n}{h}\PY{p}{,} \PY{n}{w}\PY{p}{,} \PY{l+m+mi}{4}\PY{p}{,} \PY{l+m+mi}{1}\PY{p}{]} \PY{o}{=} \PY{n}{dct\PYZus{}data}\PY{p}{[}\PY{n}{h}\PY{p}{,} \PY{n}{w}\PY{p}{,} \PY{l+m+mi}{3}\PY{p}{,} \PY{l+m+mi}{2}\PY{p}{]}
                        \PY{n}{dct\PYZus{}data}\PY{p}{[}\PY{n}{h}\PY{p}{,} \PY{n}{w}\PY{p}{,} \PY{l+m+mi}{3}\PY{p}{,} \PY{l+m+mi}{2}\PY{p}{]} \PY{o}{=} \PY{n}{tmp}
                \PY{k}{else}\PY{p}{:}
                    \PY{k}{if} \PY{n}{dct\PYZus{}data}\PY{p}{[}\PY{n}{h}\PY{p}{,} \PY{n}{w}\PY{p}{,} \PY{l+m+mi}{4}\PY{p}{,} \PY{l+m+mi}{1}\PY{p}{]} \PY{o}{\PYZgt{}} \PY{n}{dct\PYZus{}data}\PY{p}{[}\PY{n}{h}\PY{p}{,} \PY{n}{w}\PY{p}{,} \PY{l+m+mi}{3}\PY{p}{,} \PY{l+m+mi}{2}\PY{p}{]}\PY{p}{:}
                        \PY{n}{tmp} \PY{o}{=} \PY{n}{dct\PYZus{}data}\PY{p}{[}\PY{n}{h}\PY{p}{,} \PY{n}{w}\PY{p}{,} \PY{l+m+mi}{4}\PY{p}{,} \PY{l+m+mi}{1}\PY{p}{]}
                        \PY{n}{dct\PYZus{}data}\PY{p}{[}\PY{n}{h}\PY{p}{,} \PY{n}{w}\PY{p}{,} \PY{l+m+mi}{4}\PY{p}{,} \PY{l+m+mi}{1}\PY{p}{]} \PY{o}{=} \PY{n}{dct\PYZus{}data}\PY{p}{[}\PY{n}{h}\PY{p}{,} \PY{n}{w}\PY{p}{,} \PY{l+m+mi}{3}\PY{p}{,} \PY{l+m+mi}{2}\PY{p}{]}
                        \PY{n}{dct\PYZus{}data}\PY{p}{[}\PY{n}{h}\PY{p}{,} \PY{n}{w}\PY{p}{,} \PY{l+m+mi}{3}\PY{p}{,} \PY{l+m+mi}{2}\PY{p}{]} \PY{o}{=} \PY{n}{tmp}
                \PY{k}{if} \PY{n}{dct\PYZus{}data}\PY{p}{[}\PY{n}{h}\PY{p}{,} \PY{n}{w}\PY{p}{,} \PY{l+m+mi}{4}\PY{p}{,} \PY{l+m+mi}{1}\PY{p}{]} \PY{o}{\PYZlt{}} \PY{n}{dct\PYZus{}data}\PY{p}{[}\PY{n}{h}\PY{p}{,} \PY{n}{w}\PY{p}{,} \PY{l+m+mi}{3}\PY{p}{,} \PY{l+m+mi}{2}\PY{p}{]}\PY{p}{:}
                    \PY{n}{dct\PYZus{}data}\PY{p}{[}\PY{n}{h}\PY{p}{,} \PY{n}{w}\PY{p}{,} \PY{l+m+mi}{4}\PY{p}{,} \PY{l+m+mi}{1}\PY{p}{]} \PY{o}{\PYZhy{}}\PY{o}{=} \PY{n+nb+bp}{self}\PY{o}{.}\PY{n}{alpha}
                \PY{k}{else}\PY{p}{:}
                    \PY{n}{dct\PYZus{}data}\PY{p}{[}\PY{n}{h}\PY{p}{,} \PY{n}{w}\PY{p}{,} \PY{l+m+mi}{3}\PY{p}{,} \PY{l+m+mi}{2}\PY{p}{]} \PY{o}{\PYZhy{}}\PY{o}{=} \PY{n+nb+bp}{self}\PY{o}{.}\PY{n}{alpha}
        \PY{k}{return} \PY{n}{dct\PYZus{}data}

    \PY{k}{def} \PY{n+nf}{idct\PYZus{}embed}\PY{p}{(}\PY{n+nb+bp}{self}\PY{p}{,} \PY{n}{dct\PYZus{}data}\PY{p}{)}\PY{p}{:}
\PY{+w}{        }\PY{l+s+sd}{\PYZdq{}\PYZdq{}\PYZdq{}}
\PY{l+s+sd}{        进行对dct矩阵进行idct变换,完成从频域到空域的变换}
\PY{l+s+sd}{        :param dct\PYZus{}data: 频域数据}
\PY{l+s+sd}{        :return: 空域数据}
\PY{l+s+sd}{        \PYZdq{}\PYZdq{}\PYZdq{}}
        \PY{n}{row} \PY{o}{=} \PY{k+kc}{None}
        \PY{n}{result} \PY{o}{=} \PY{k+kc}{None}
        \PY{n}{h}\PY{p}{,} \PY{n}{w} \PY{o}{=} \PY{n}{dct\PYZus{}data}\PY{o}{.}\PY{n}{shape}\PY{p}{[}\PY{l+m+mi}{0}\PY{p}{]}\PY{p}{,} \PY{n}{dct\PYZus{}data}\PY{o}{.}\PY{n}{shape}\PY{p}{[}\PY{l+m+mi}{1}\PY{p}{]}
        \PY{k}{for} \PY{n}{i} \PY{o+ow}{in} \PY{n+nb}{range}\PY{p}{(}\PY{n}{h}\PY{p}{)}\PY{p}{:}
            \PY{k}{for} \PY{n}{j} \PY{o+ow}{in} \PY{n+nb}{range}\PY{p}{(}\PY{n}{w}\PY{p}{)}\PY{p}{:}
                \PY{n}{block} \PY{o}{=} \PY{n}{cv2}\PY{o}{.}\PY{n}{idct}\PY{p}{(}\PY{n}{dct\PYZus{}data}\PY{p}{[}\PY{n}{i}\PY{p}{,} \PY{n}{j}\PY{p}{,} \PY{o}{.}\PY{o}{.}\PY{o}{.}\PY{p}{]}\PY{p}{)}
                \PY{n}{row} \PY{o}{=} \PY{n}{block} \PY{k}{if} \PY{n}{j} \PY{o}{==} \PY{l+m+mi}{0} \PY{k}{else} \PY{n}{np}\PY{o}{.}\PY{n}{hstack}\PY{p}{(}\PY{p}{(}\PY{n}{row}\PY{p}{,} \PY{n}{block}\PY{p}{)}\PY{p}{)}
            \PY{n}{result} \PY{o}{=} \PY{n}{row} \PY{k}{if} \PY{n}{i} \PY{o}{==} \PY{l+m+mi}{0} \PY{k}{else} \PY{n}{np}\PY{o}{.}\PY{n}{vstack}\PY{p}{(}\PY{p}{(}\PY{n}{result}\PY{p}{,} \PY{n}{row}\PY{p}{)}\PY{p}{)}
        \PY{k}{return} \PY{n}{result}\PY{o}{.}\PY{n}{astype}\PY{p}{(}\PY{n}{np}\PY{o}{.}\PY{n}{uint8}\PY{p}{)}

    \PY{k}{def} \PY{n+nf}{dct\PYZus{}extract}\PY{p}{(}\PY{n+nb+bp}{self}\PY{p}{,} \PY{n}{synthesis}\PY{p}{,} \PY{n}{watermark\PYZus{}size}\PY{p}{)}\PY{p}{:}
\PY{+w}{        }\PY{l+s+sd}{\PYZdq{}\PYZdq{}\PYZdq{}}
\PY{l+s+sd}{        从嵌入水印的图像中提取水印}
\PY{l+s+sd}{        :param synthesis: 嵌入水印的空域图像}
\PY{l+s+sd}{        :param watermark\PYZus{}size: 水印大小}
\PY{l+s+sd}{        :return: 提取的空域水印}
\PY{l+s+sd}{        \PYZdq{}\PYZdq{}\PYZdq{}}
        \PY{n}{w\PYZus{}h}\PY{p}{,} \PY{n}{w\PYZus{}w} \PY{o}{=} \PY{n}{watermark\PYZus{}size}
        \PY{n}{recover\PYZus{}watermark} \PY{o}{=} \PY{n}{np}\PY{o}{.}\PY{n}{zeros}\PY{p}{(}\PY{n}{shape}\PY{o}{=}\PY{n}{watermark\PYZus{}size}\PY{p}{)}
        \PY{n}{synthesis\PYZus{}dct\PYZus{}blocks} \PY{o}{=} \PY{n+nb+bp}{self}\PY{o}{.}\PY{n}{dct\PYZus{}blkproc}\PY{p}{(}\PY{n}{background}\PY{o}{=}\PY{n}{synthesis}\PY{p}{)}
        \PY{k}{for} \PY{n}{h} \PY{o+ow}{in} \PY{n+nb}{range}\PY{p}{(}\PY{n}{w\PYZus{}h}\PY{p}{)}\PY{p}{:}
            \PY{k}{for} \PY{n}{w} \PY{o+ow}{in} \PY{n+nb}{range}\PY{p}{(}\PY{n}{w\PYZus{}w}\PY{p}{)}\PY{p}{:}

                \PY{k}{if} \PY{n}{synthesis\PYZus{}dct\PYZus{}blocks}\PY{p}{[}\PY{n}{h}\PY{p}{,} \PY{n}{w}\PY{p}{,} \PY{l+m+mi}{4}\PY{p}{,} \PY{l+m+mi}{1}\PY{p}{]} \PY{o}{\PYZlt{}} \PY{n}{synthesis\PYZus{}dct\PYZus{}blocks}\PY{p}{[}\PY{n}{h}\PY{p}{,} \PY{n}{w}\PY{p}{,} \PY{l+m+mi}{3}\PY{p}{,} \PY{l+m+mi}{2}\PY{p}{]}\PY{p}{:}
                    \PY{n}{recover\PYZus{}watermark}\PY{p}{[}\PY{n}{h}\PY{p}{,} \PY{n}{w}\PY{p}{]} \PY{o}{=} \PY{l+m+mi}{1}
                \PY{k}{else}\PY{p}{:}
                    \PY{n}{recover\PYZus{}watermark}\PY{p}{[}\PY{n}{h}\PY{p}{,} \PY{n}{w}\PY{p}{]} \PY{o}{=} \PY{l+m+mi}{0}
        \PY{k}{return} \PY{n}{recover\PYZus{}watermark}
\end{Verbatim}
\end{tcolorbox}

    \begin{tcolorbox}[breakable, size=fbox, boxrule=1pt, pad at break*=1mm,colback=cellbackground, colframe=cellborder]
\prompt{In}{incolor}{12}{\boxspacing}
\begin{Verbatim}[commandchars=\\\{\}]
\PY{k}{if} \PY{n+nv+vm}{\PYZus{}\PYZus{}name\PYZus{}\PYZus{}} \PY{o}{==} \PY{l+s+s1}{\PYZsq{}}\PY{l+s+s1}{\PYZus{}\PYZus{}main\PYZus{}\PYZus{}}\PY{l+s+s1}{\PYZsq{}}\PY{p}{:}
    \PY{n}{root} \PY{o}{=} \PY{l+s+s2}{\PYZdq{}}\PY{l+s+s2}{..}\PY{l+s+s2}{\PYZdq{}}

    \PY{c+c1}{\PYZsh{} 0. 超参数设置}
    \PY{n}{alpha} \PY{o}{=} \PY{l+m+mi}{100}  \PY{c+c1}{\PYZsh{} 尺度控制因子,控制水印添加强度,决定频域系数被修改的幅度}
    \PY{n}{blocksize} \PY{o}{=} \PY{l+m+mi}{8}  \PY{c+c1}{\PYZsh{} 分块大小}

    \PY{c+c1}{\PYZsh{} 1. 数据读取}

    \PY{c+c1}{\PYZsh{} watermak}
    \PY{n}{watermark} \PY{o}{=} \PY{n}{cv2}\PY{o}{.}\PY{n}{imread}\PY{p}{(}\PY{l+s+sa}{r}\PY{l+s+s2}{\PYZdq{}}\PY{l+s+s2}{./watermark\PYZus{}resize.png}\PY{l+s+s2}{\PYZdq{}}\PY{o}{.}\PY{n}{format}\PY{p}{(}\PY{n}{root}\PY{p}{)}\PY{p}{,} \PY{n}{cv2}\PY{o}{.}\PY{n}{IMREAD\PYZus{}GRAYSCALE}\PY{p}{)}
    \PY{n}{watermark} \PY{o}{=} \PY{n}{np}\PY{o}{.}\PY{n}{where}\PY{p}{(}\PY{n}{watermark} \PY{o}{\PYZlt{}} \PY{n}{np}\PY{o}{.}\PY{n}{mean}\PY{p}{(}\PY{n}{watermark}\PY{p}{)}\PY{p}{,} \PY{l+m+mi}{0}\PY{p}{,} \PY{l+m+mi}{1}\PY{p}{)}  \PY{c+c1}{\PYZsh{} watermark进行(归一化的)二值化}
    \PY{n}{background} \PY{o}{=} \PY{n}{cv2}\PY{o}{.}\PY{n}{imread}\PY{p}{(}\PY{l+s+sa}{r}\PY{l+s+s2}{\PYZdq{}}\PY{l+s+s2}{./Lena\PYZus{}256.bmp}\PY{l+s+s2}{\PYZdq{}}\PY{o}{.}\PY{n}{format}\PY{p}{(}\PY{n}{root}\PY{p}{)}\PY{p}{,} \PY{n}{cv2}\PY{o}{.}\PY{n}{IMREAD\PYZus{}GRAYSCALE}\PY{p}{)}

    \PY{c+c1}{\PYZsh{} 2. 初始化DCT算法}
    \PY{n}{dct\PYZus{}emb} \PY{o}{=} \PY{n}{DCT\PYZus{}Embed}\PY{p}{(}\PY{n}{background}\PY{o}{=}\PY{n}{background}\PY{p}{,} \PY{n}{watermark}\PY{o}{=}\PY{n}{watermark}\PY{p}{,} \PY{n}{block\PYZus{}size}\PY{o}{=}\PY{n}{blocksize}\PY{p}{,} \PY{n}{alpha}\PY{o}{=}\PY{n}{alpha}\PY{p}{)}

    \PY{c+c1}{\PYZsh{} 3. 进行分块与DCT变换}
    \PY{n}{background\PYZus{}dct\PYZus{}blocks} \PY{o}{=} \PY{n}{dct\PYZus{}emb}\PY{o}{.}\PY{n}{dct\PYZus{}blkproc}\PY{p}{(}\PY{n}{background}\PY{o}{=}\PY{n}{background}\PY{p}{)}  \PY{c+c1}{\PYZsh{} 得到分块的DCTblocks}

    \PY{c+c1}{\PYZsh{} 4. 嵌入水印图像}
    \PY{n}{embed\PYZus{}watermark\PYZus{}blocks} \PY{o}{=} \PY{n}{dct\PYZus{}emb}\PY{o}{.}\PY{n}{dct\PYZus{}embed}\PY{p}{(}\PY{n}{dct\PYZus{}data}\PY{o}{=}\PY{n}{background\PYZus{}dct\PYZus{}blocks}\PY{p}{,} \PY{n}{watermark}\PY{o}{=}\PY{n}{watermark}\PY{p}{)}  \PY{c+c1}{\PYZsh{} 在dct块中嵌入水印图像}

    \PY{c+c1}{\PYZsh{} 5. 将图像转换为空域形式}
    \PY{n}{synthesis} \PY{o}{=} \PY{n}{dct\PYZus{}emb}\PY{o}{.}\PY{n}{idct\PYZus{}embed}\PY{p}{(}\PY{n}{dct\PYZus{}data}\PY{o}{=}\PY{n}{embed\PYZus{}watermark\PYZus{}blocks}\PY{p}{)}  \PY{c+c1}{\PYZsh{} idct变换得到空域图像}
    \PY{n+nb}{print}\PY{p}{(}\PY{l+s+s2}{\PYZdq{}}\PY{l+s+s2}{PSNR}\PY{l+s+s2}{\PYZdq{}}\PY{p}{,}\PY{n}{calculate\PYZus{}psnr}\PY{p}{(}\PY{n}{background}\PY{p}{,} \PY{n}{synthesis}\PY{p}{)}\PY{p}{)}

    \PY{c+c1}{\PYZsh{} 6. 提取水印}
    \PY{n}{extract\PYZus{}watermark} \PY{o}{=} \PY{n}{dct\PYZus{}emb}\PY{o}{.}\PY{n}{dct\PYZus{}extract}\PY{p}{(}\PY{n}{synthesis}\PY{o}{=}\PY{n}{synthesis}\PY{p}{,} \PY{n}{watermark\PYZus{}size}\PY{o}{=}\PY{n}{watermark}\PY{o}{.}\PY{n}{shape}\PY{p}{)} \PY{o}{*} \PY{l+m+mi}{255}
    \PY{n}{extract\PYZus{}watermark}\PY{o}{.}\PY{n}{astype}\PY{p}{(}\PY{n}{np}\PY{o}{.}\PY{n}{uint8}\PY{p}{)}
    \PY{c+c1}{\PYZsh{} 7. 可视化处理}
    \PY{n}{images} \PY{o}{=} \PY{p}{[}\PY{n}{background}\PY{p}{,} \PY{n}{watermark}\PY{p}{,} \PY{n}{synthesis}\PY{p}{,} \PY{n}{extract\PYZus{}watermark}\PY{p}{]}
    \PY{n}{titles} \PY{o}{=} \PY{p}{[}\PY{l+s+s2}{\PYZdq{}}\PY{l+s+s2}{background}\PY{l+s+s2}{\PYZdq{}}\PY{p}{,} \PY{l+s+s2}{\PYZdq{}}\PY{l+s+s2}{watermark}\PY{l+s+s2}{\PYZdq{}}\PY{p}{,} \PY{l+s+s2}{\PYZdq{}}\PY{l+s+s2}{systhesis}\PY{l+s+s2}{\PYZdq{}}\PY{p}{,} \PY{l+s+s2}{\PYZdq{}}\PY{l+s+s2}{extract}\PY{l+s+s2}{\PYZdq{}}\PY{p}{]}
    \PY{k}{for} \PY{n}{i} \PY{o+ow}{in} \PY{n+nb}{range}\PY{p}{(}\PY{l+m+mi}{4}\PY{p}{)}\PY{p}{:}
        \PY{n}{plt}\PY{o}{.}\PY{n}{subplot}\PY{p}{(}\PY{l+m+mi}{2}\PY{p}{,} \PY{l+m+mi}{2}\PY{p}{,} \PY{n}{i} \PY{o}{+} \PY{l+m+mi}{1}\PY{p}{)}
        \PY{k}{if} \PY{n}{i} \PY{o}{\PYZpc{}} \PY{l+m+mi}{2}\PY{p}{:}
            \PY{n}{plt}\PY{o}{.}\PY{n}{imshow}\PY{p}{(}\PY{n}{images}\PY{p}{[}\PY{n}{i}\PY{p}{]}\PY{p}{,} \PY{n}{cmap}\PY{o}{=}\PY{n}{plt}\PY{o}{.}\PY{n}{cm}\PY{o}{.}\PY{n}{gray}\PY{p}{)}
        \PY{k}{else}\PY{p}{:}
            \PY{n}{plt}\PY{o}{.}\PY{n}{imshow}\PY{p}{(}\PY{n}{images}\PY{p}{[}\PY{n}{i}\PY{p}{]}\PY{p}{)}
        \PY{n}{plt}\PY{o}{.}\PY{n}{title}\PY{p}{(}\PY{n}{titles}\PY{p}{[}\PY{n}{i}\PY{p}{]}\PY{p}{)}
        \PY{n}{plt}\PY{o}{.}\PY{n}{axis}\PY{p}{(}\PY{l+s+s2}{\PYZdq{}}\PY{l+s+s2}{off}\PY{l+s+s2}{\PYZdq{}}\PY{p}{)}
    \PY{n}{plt}\PY{o}{.}\PY{n}{show}\PY{p}{(}\PY{p}{)}
\end{Verbatim}
\end{tcolorbox}

    \begin{Verbatim}[commandchars=\\\{\}]
PSNR 28.804966484513063
    \end{Verbatim}

    \begin{center}
    \adjustimage{max size={0.9\linewidth}{0.9\paperheight}}{output_26_1.png}
    \end{center}
    { \hspace*{\fill} \\}
    
    因为需要通过比较变换后的两个DCT系数来完成信息的隐藏,所以在传递秘密信息前,通信的双方就必须对要比较的两个位置达成一致。对于每一个\(8\times8\)图像块,共有64个系数,为了隐藏信息的不可见性,要求在其中选择两个恰当的系数\(a\),\(b\),使得这两个位置上的数据在图像经过处理后差别不大。因此,选择DCT系数中的中频系数,可以兼顾信息隐藏的不可见性与鲁棒性。如果选择低频系数,由于其所相应的能量过大,秘密消息的不可见性差,如果选择高频系数,则能量最低,很容易被篡改,鲁棒性差。中频系数的频率适中,既不会太容易被篡改,保证了鲁棒性,也不会能量太高,破坏不可见性。综上,选择\((5,2)\)和\((4,3)\)这一对系数。但由于这样的一对系数大小相差很少,往往难以保证隐秘图像在保存、信道上传输以及提取信息时再次被读取等过程中不发生变化。所以,再引入一个控制量\(\alpha\)对系数差值进行放大。这样即使在变换过程中系数的值有轻微的改变,也不会影响编码的正确性。当$\alpha=0$时,即不设置$\alpha$时,可以发现信息提取有误。

    \begin{tcolorbox}[breakable, size=fbox, boxrule=1pt, pad at break*=1mm,colback=cellbackground, colframe=cellborder]
\prompt{In}{incolor}{13}{\boxspacing}
\begin{Verbatim}[commandchars=\\\{\}]
\PY{k}{if} \PY{n+nv+vm}{\PYZus{}\PYZus{}name\PYZus{}\PYZus{}} \PY{o}{==} \PY{l+s+s1}{\PYZsq{}}\PY{l+s+s1}{\PYZus{}\PYZus{}main\PYZus{}\PYZus{}}\PY{l+s+s1}{\PYZsq{}}\PY{p}{:}
    \PY{n}{root} \PY{o}{=} \PY{l+s+s2}{\PYZdq{}}\PY{l+s+s2}{..}\PY{l+s+s2}{\PYZdq{}}

    \PY{c+c1}{\PYZsh{} 0. 超参数设置}
    \PY{n}{alpha} \PY{o}{=} \PY{l+m+mi}{0}  \PY{c+c1}{\PYZsh{} 尺度控制因子,控制水印添加强度,决定频域系数被修改的幅度}
    \PY{n}{blocksize} \PY{o}{=} \PY{l+m+mi}{8}  \PY{c+c1}{\PYZsh{} 分块大小}

    \PY{c+c1}{\PYZsh{} 1. 数据读取}

    \PY{c+c1}{\PYZsh{} watermak}
    \PY{n}{watermark} \PY{o}{=} \PY{n}{cv2}\PY{o}{.}\PY{n}{imread}\PY{p}{(}\PY{l+s+sa}{r}\PY{l+s+s2}{\PYZdq{}}\PY{l+s+s2}{./watermark\PYZus{}resize.png}\PY{l+s+s2}{\PYZdq{}}\PY{o}{.}\PY{n}{format}\PY{p}{(}\PY{n}{root}\PY{p}{)}\PY{p}{,} \PY{n}{cv2}\PY{o}{.}\PY{n}{IMREAD\PYZus{}GRAYSCALE}\PY{p}{)}
    \PY{n}{watermark} \PY{o}{=} \PY{n}{np}\PY{o}{.}\PY{n}{where}\PY{p}{(}\PY{n}{watermark} \PY{o}{\PYZlt{}} \PY{n}{np}\PY{o}{.}\PY{n}{mean}\PY{p}{(}\PY{n}{watermark}\PY{p}{)}\PY{p}{,} \PY{l+m+mi}{0}\PY{p}{,} \PY{l+m+mi}{1}\PY{p}{)}  \PY{c+c1}{\PYZsh{} watermark进行(归一化的)二值化}
    \PY{n}{background} \PY{o}{=} \PY{n}{cv2}\PY{o}{.}\PY{n}{imread}\PY{p}{(}\PY{l+s+sa}{r}\PY{l+s+s2}{\PYZdq{}}\PY{l+s+s2}{./Lena\PYZus{}256.bmp}\PY{l+s+s2}{\PYZdq{}}\PY{o}{.}\PY{n}{format}\PY{p}{(}\PY{n}{root}\PY{p}{)}\PY{p}{,} \PY{n}{cv2}\PY{o}{.}\PY{n}{IMREAD\PYZus{}GRAYSCALE}\PY{p}{)}

    \PY{c+c1}{\PYZsh{} 2. 初始化DCT算法}
    \PY{n}{dct\PYZus{}emb} \PY{o}{=} \PY{n}{DCT\PYZus{}Embed}\PY{p}{(}\PY{n}{background}\PY{o}{=}\PY{n}{background}\PY{p}{,} \PY{n}{watermark}\PY{o}{=}\PY{n}{watermark}\PY{p}{,} \PY{n}{block\PYZus{}size}\PY{o}{=}\PY{n}{blocksize}\PY{p}{,} \PY{n}{alpha}\PY{o}{=}\PY{n}{alpha}\PY{p}{)}

    \PY{c+c1}{\PYZsh{} 3. 进行分块与DCT变换}
    \PY{n}{background\PYZus{}dct\PYZus{}blocks} \PY{o}{=} \PY{n}{dct\PYZus{}emb}\PY{o}{.}\PY{n}{dct\PYZus{}blkproc}\PY{p}{(}\PY{n}{background}\PY{o}{=}\PY{n}{background}\PY{p}{)}  \PY{c+c1}{\PYZsh{} 得到分块的DCTblocks}

    \PY{c+c1}{\PYZsh{} 4. 嵌入水印图像}
    \PY{n}{embed\PYZus{}watermark\PYZus{}blocks} \PY{o}{=} \PY{n}{dct\PYZus{}emb}\PY{o}{.}\PY{n}{dct\PYZus{}embed}\PY{p}{(}\PY{n}{dct\PYZus{}data}\PY{o}{=}\PY{n}{background\PYZus{}dct\PYZus{}blocks}\PY{p}{,} \PY{n}{watermark}\PY{o}{=}\PY{n}{watermark}\PY{p}{)}  \PY{c+c1}{\PYZsh{} 在dct块中嵌入水印图像}

    \PY{c+c1}{\PYZsh{} 5. 将图像转换为空域形式}
    \PY{n}{synthesis} \PY{o}{=} \PY{n}{dct\PYZus{}emb}\PY{o}{.}\PY{n}{idct\PYZus{}embed}\PY{p}{(}\PY{n}{dct\PYZus{}data}\PY{o}{=}\PY{n}{embed\PYZus{}watermark\PYZus{}blocks}\PY{p}{)}  \PY{c+c1}{\PYZsh{} idct变换得到空域图像}
    \PY{n+nb}{print}\PY{p}{(}\PY{l+s+s2}{\PYZdq{}}\PY{l+s+s2}{PSNR}\PY{l+s+s2}{\PYZdq{}}\PY{p}{,}\PY{n}{calculate\PYZus{}psnr}\PY{p}{(}\PY{n}{background}\PY{p}{,} \PY{n}{synthesis}\PY{p}{)}\PY{p}{)}

    \PY{c+c1}{\PYZsh{} 6. 提取水印}
    \PY{n}{extract\PYZus{}watermark} \PY{o}{=} \PY{n}{dct\PYZus{}emb}\PY{o}{.}\PY{n}{dct\PYZus{}extract}\PY{p}{(}\PY{n}{synthesis}\PY{o}{=}\PY{n}{synthesis}\PY{p}{,} \PY{n}{watermark\PYZus{}size}\PY{o}{=}\PY{n}{watermark}\PY{o}{.}\PY{n}{shape}\PY{p}{)} \PY{o}{*} \PY{l+m+mi}{255}
    \PY{n}{extract\PYZus{}watermark}\PY{o}{.}\PY{n}{astype}\PY{p}{(}\PY{n}{np}\PY{o}{.}\PY{n}{uint8}\PY{p}{)}
    \PY{c+c1}{\PYZsh{} 7. 可视化处理}
    \PY{n}{images} \PY{o}{=} \PY{p}{[}\PY{n}{background}\PY{p}{,} \PY{n}{watermark}\PY{p}{,} \PY{n}{synthesis}\PY{p}{,} \PY{n}{extract\PYZus{}watermark}\PY{p}{]}
    \PY{n}{titles} \PY{o}{=} \PY{p}{[}\PY{l+s+s2}{\PYZdq{}}\PY{l+s+s2}{background}\PY{l+s+s2}{\PYZdq{}}\PY{p}{,} \PY{l+s+s2}{\PYZdq{}}\PY{l+s+s2}{watermark}\PY{l+s+s2}{\PYZdq{}}\PY{p}{,} \PY{l+s+s2}{\PYZdq{}}\PY{l+s+s2}{systhesis}\PY{l+s+s2}{\PYZdq{}}\PY{p}{,} \PY{l+s+s2}{\PYZdq{}}\PY{l+s+s2}{extract}\PY{l+s+s2}{\PYZdq{}}\PY{p}{]}
    \PY{k}{for} \PY{n}{i} \PY{o+ow}{in} \PY{n+nb}{range}\PY{p}{(}\PY{l+m+mi}{4}\PY{p}{)}\PY{p}{:}
        \PY{n}{plt}\PY{o}{.}\PY{n}{subplot}\PY{p}{(}\PY{l+m+mi}{2}\PY{p}{,} \PY{l+m+mi}{2}\PY{p}{,} \PY{n}{i} \PY{o}{+} \PY{l+m+mi}{1}\PY{p}{)}
        \PY{k}{if} \PY{n}{i} \PY{o}{\PYZpc{}} \PY{l+m+mi}{2}\PY{p}{:}
            \PY{n}{plt}\PY{o}{.}\PY{n}{imshow}\PY{p}{(}\PY{n}{images}\PY{p}{[}\PY{n}{i}\PY{p}{]}\PY{p}{,} \PY{n}{cmap}\PY{o}{=}\PY{n}{plt}\PY{o}{.}\PY{n}{cm}\PY{o}{.}\PY{n}{gray}\PY{p}{)}
        \PY{k}{else}\PY{p}{:}
            \PY{n}{plt}\PY{o}{.}\PY{n}{imshow}\PY{p}{(}\PY{n}{images}\PY{p}{[}\PY{n}{i}\PY{p}{]}\PY{p}{)}
        \PY{n}{plt}\PY{o}{.}\PY{n}{title}\PY{p}{(}\PY{n}{titles}\PY{p}{[}\PY{n}{i}\PY{p}{]}\PY{p}{)}
        \PY{n}{plt}\PY{o}{.}\PY{n}{axis}\PY{p}{(}\PY{l+s+s2}{\PYZdq{}}\PY{l+s+s2}{off}\PY{l+s+s2}{\PYZdq{}}\PY{p}{)}
    \PY{n}{plt}\PY{o}{.}\PY{n}{show}\PY{p}{(}\PY{p}{)}
\end{Verbatim}
\end{tcolorbox}

    \begin{Verbatim}[commandchars=\\\{\}]
PSNR 42.222951891600744
    \end{Verbatim}

    \begin{center}
    \adjustimage{max size={0.9\linewidth}{0.9\paperheight}}{output_28_1.png}
    \end{center}
    { \hspace*{\fill} \\}
    

    % Add a bibliography block to the postdoc
    
    
    
\end{document}
